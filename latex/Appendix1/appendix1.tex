%!TEX root = ../thesis.tex
% ******************************* Thesis Appendix A **********************************
\chapter[Βαθμονόμηση μετρητή παροχής]{Βαθμονόμηση ογκομετριτή θετικής μετατόπισης}\label{app:calibration}

% ******************************* Nomenclature ****************************************

\nomenclature[z-\ITU]{\ITU}{İstanbul Teknik Üniversitesi}
\nomenclature[z-err]{err}{Root mean square}
\nomenclature[x-$\iint$]{$\iint$}{Διπλό ολοκλήρωμα}
\nomenclature[s-μ.π.]{μ.π.}{Μετρητής παροχής}
\nomenclature[s-θ.ν.]{θ.ν.}{Θερμό νήμα}


\begin{refsection}

\noindent Οι μετρητές θετικής μετατόπισης απαρτίζονται από κινούμενα μέρη που ορίζουν έναν γνωστό, μετρούμενο όγκο, γεμάτο με ρευστό. Τα κινούμενα μέρη μετατοπίζονται (ή περιστρέφονται) με τη δράση του κινούμενου ρευστού και ο συνολικός όγκος που έχει περάσει μπορεί να διαβαστεί από μονάδα αναλογικής απεικόνισης στο πλάι του μετρητή. Αυτή η μέθοδος μέτρησης είναι κοινή στους μετρητές νερού, βενζίνης και φυσικού αερίου. Για τον προσδιορισμό ροής όγκου, η μέτρηση όγκου από τη συσκευή πρέπει να χρησιμοποιηθεί σε συνδυασμό με χρονομέτρη, έτσι ώστε:

\begin{align}\label{eq:vol}
\dot{\volume} = \frac{\volume}{t}
\end{align}
 
\noindent Στην παρούσα έρευνα χρησιμοποιήθηκε περιστροφικός μετρητής αερίου θετικής μετατόπισης, γνωστός και ως μετρητής Roots \cite{2000_Baker_BOOK_CHAPTER}. Στο \prettyref{fig:flowmet} φαίνεται ένα σκαρίφημα τυπικό αυτού του είδους μετρητή. Δύο στροφείς περιστρέφονται μέσα σε ένα οβάλ περίγραμμα σώματος ούτως ώστε η διαρροή να περιοριστεί στο ελάχιστο, και το αέριο να μεταφερθεί στην έξοδο. Οι δύο στροφείς είναι τοποθετημένοι με πολύ μικρές ανοχές και, ως εκ τούτου, είναι απαραίτητο το αέριο να είναι απαλλαγμένο από μικροσωματίδια.

\begin{figure} 
\centering    
\subimport{Appendix1/Figs/pdftex}{flowmeter.pdf_tex}
\caption{Περιστροφικός μετρητής θετικής μετατόπισης \cite{2000_Baker_BOOK_CHAPTER}: (α) Τομή μετρητή (β) Ενδεικτική απόδοση μετρητή}
\label{fig:flowmet}
\end{figure}

H αβεβαιότητα μπορεί να κυμανθεί μέχρι και το 0.2\% (95\%) της πραγματικής ροής. Ωστόσο, η ακρίβεια τους μπορεί να υποβαθμιστεί λόγω παλλόμενης ροής που προκύπτει λόγω σχεδιασμού \cite{1982_Dijstelbergen_BOOK_CHAPTER}, καθώς επίσης και από βίαια μεταβαλλόμενους ρυθμούς ροής.

\section*{Σχετική βαθμονόμηση}

\noindent Αν και θα μπορούσαμε να λάβουμε την αβεβαιότητα του μετρητικού όπως συνιστάται από τη βιβλιογραφία \parencites{2000_Baker_BOOK_CHAPTER}{1982_Dijstelbergen_BOOK_CHAPTER}{2011_Figliola_BOOK}, η πολυετή του χρήση επιβάλλει τη διεξαγωγή βαθμονόμησης. Αλλοίωση των χαρακτηριστικών του οργάνου λόγω γήρανσης, παρουσίας σκόνης στους στροφείς, είναι μερικοί από τους πολλούς παράγοντες που μπορούν να επηρεάσουν το τελικό αποτέλεσμα της μέτρησης.
Χρησιμοποιήθηκε λοιπόν μετρητικό όργανο ανεμομετρίας θερμού νήματος (hot-wire) του \ITU. Λόγω της μεγάλη ακρίβειας αυτού του μετρητικού, θεωρήθηκε δόκιμο να χρησιμοποιηθεί ως περιστασιακό πρότυπο εργασίας για τη βαθμονόμηση του περιστροφικού μετρητή αερίου.

Η διαδικασία που θα ακολουθήσει αποκλίνει λίγο από την τυπική της βαθμονόμησης: σχεδιασμός καμπύλης βαθμονόμησης (και της αβεβαιότητας αυτής) στον χώρο που πρόκειται να χρησιμοποιηθεί το μετρητικό, και υπό τις συνθήκες μετρήσεων στις οποίες θα διεξαχθεί το πείραμα. Δεν θα σχεδιαστεί καμία καμπύλη βαθμονόμησης. Θα συγκριθούν μονάχα οι σχετικές αποκλίσεις μεταξύ μετρητικού (μετρητή παροχής) και προτύπου (θερμό νήμα), για διάφορα εύρη, και θα κριθεί εντέλει κατά πόσο έχει νόημα η χρήση του συγκεκριμένου μετρητή θετικής μετατόπισης.

Η εκτίμηση του σφάλματος θα γίνει υπολογίζοντας το ποσοστιαίο σφάλμα μεταξύ των δύο μετρητικών από:

\begin{align}\label{eq:voler}
\dot{\volume}_{\text{σχ.}} = 100 \, \frac{\dot{\volume}_{\text{μ.π.}} - \dot{\volume}_{\text{θ.ν.}}}{\dot{\volume}_{\text{θ.ν.}}}
\end{align}

\noindent και για την αναγωγή του σε ολόκληρο το εύρος παροχής, υπολογίζεται το σφάλμα Root-mean-square (err) όπως ορίζεται από τους \citeauthor{1962_Kenney_BOOK_CHAPTER} \cite{1962_Kenney_BOOK_CHAPTER},

\begin{align}\label{eq:rms}
\text{err}_{\text{εύρος}} &= \left[\frac{1}{N_{\text{μ.π.}}}\sum_{n = 1}^{N_{\text{μ.π.}}}\frac{\left(\dot{\volume}^{\text{μ.π.}}_{n} - \dot{\volume}^{\text{θ.ν.}}_{n}\right)^2}{\left(\dot{\volume}^{\text{θ.ν.}}_{n}\right)^2}100\right]^{1/2}\nonumber\\
&= \left[\frac{1}{N_{\text{μ.π.}}}\sum_{n = 1}^{N_{\text{μ.π.}}}\left(\frac{\dot{\volume}^{\text{μ.π.}}_{n} - \dot{\volume}^{\text{θ.ν.}}_{n}}{\dot{\volume}^{\text{θ.ν.}}_{n}}100\right)^2 \right]^{1/2}\nonumber\\
&= \left[\frac{1}{N_{\text{μ.π.}}}\sum_{n = 1}^{N_{\text{μ.π.}}} \left(\dot{\volume}^{\text{σχ.}}_{n} \right)^2 \right]^{1/2}\\
&= \left[\frac{1}{N_{\text{μ.π.}}} \, \sum_{n = 1}^{N_{\text{μ.π.}}} \, \left(\dot{\volume}_{n}^{\text{σχ.}}\right)^2 \right]^{1/2}\nonumber
\end{align}

\noindent όπου μ.ρ. και θ.ν. είναι οι μετρήσεις του μετρητή παροχής και θερμικού νήματος αντίστοιχα, και το εύρος αφορά το πλαίσιο μετρήσεων στο οποίο εφαρμόστηκε.

Οι μετρήσεις των ταχυτήτων για τη βαθμονόμηση του ογκομετρητή (δηλαδή οι ταχύτητες αναφοράς) έγιναν βάσει προτεινόμενης μεθοδολογίας από τους \citeauthor{2011_Figliola_BOOK} \cite[σελ. 425-426]{2011_Figliola_BOOK}. Η βαθμονόμηση πραγματοποιήθηκε σε διάφορες μεταβολές της παροχής/ταχύτητας, χωρίς έντονη διακύμανση, λαμβάνοντας υπόψη την ανώτερη τιμή μέτρησης που μπορεί να λάβει το θερμό νήμα. 

Η διαδικασία βαθμονόμησης που πραγματοποιήθηκε είχε ως εξής: 

\begin{enumerate}
\item Μετρήσεις σημειακών ταχυτήτων με θερμό νήμα κατά μήκος, και περί, της διατομής. 
\item Απομάκρυνση του θερμού νήματος από τη διάταξη και τοποθέτηση του ογκομετρητή παροχής. 
\item Λήψη δεδομένων από τον ογκομετρητή στις ίδιες συνθήκες ροής (ίδια ισχύς ανεμιστήρα) με αυτές του θερμικού νήματος. Τα δεδομένα λαμβάνονται υπό μορφή τάσης μέσω ηλεκτρονικής ένδειξης που φέρει το θερμικό ανεμόμετρο. 
\item Επεξεργασία δεδομένων, εξαγωγή ποσοστιαίας απόκλισης μεταξύ των μετρητικών και υπολογισμός του σφάλματος Root-mean-square.
\end{enumerate}

Λόγω πιθανών διακυμάνσεων, οι τιμές αναφοράς του θερμικού ανεμομέτρου, πρέπει να είναι οι χρονικά μέσες τιμές για ένα παρατεταμένο χρονικό διάστημα (γύρω στα 10-30s) \cite{1995_Bruun_BOOK}. Σε βαθμονομήσεις που η λήψη των μεγεθών (V) γίνεται με πολύμετρο, η παραπάνω συνθήκη ικανοποιείται με παρατήρηση των διακυμάνσεων, ενώ σε αυτές που τα δεδομένα λαμβάνονται μέσω προσαρτημένων υπολογιστικών συστημάτων (όπως στην προκείμενη περίπτωση) υπολογίζεται η μέση τιμή τους.
 
Να σημειωθεί ότι ο προσανατολισμός του θερμού νήματος, σε σχέση με τη ροή, αποτελεί παράγοντα καθοριστικής σημασίας για τη μέτρηση. Τα αισθητήρια πρέπει να τοποθετούνται εντός ροής με τον άξονά τους και κάθετα στην διεύθυνση της ροής. Στην παρούσα βαθμονόμηση, οι μετρήσεις ελήφθησαν στην έξοδο της ροής. Το θερμό νήμα τοποθετήθηκε, σε κάθε περίσταση, πλησίον της διατομής εξόδου, ούτως ώστε η θεώρηση \enquote{τοποθέτησης εντός της ροής} να έχει κάποια δόση αλήθειας.

\section*{Παράδειγμα υπολογισμού}

\noindent Οι οριακές συνθήκες που διέπουν όλες τις μετρήσεις, από κάθε όργανο μέτρησης, θεωρούνται ίδιες με αυτές που ελήφθησαν κατά τη διάρκεια του πειράματος, δηλαδή σταθερή, μόνιμη και ασυμπίεστη ροή. Το παράδειγμα που ακολουθεί αφορά τη διάταξη περιδινούμενης ροής, με 4 γενέτειρες, και κλίσης $45^o$.

Τέθηκε η πηγή τάσης, που κινεί τον ανεμιστηράκι αναρρόφησης αέρα της διάταξης, σε σταθερή τιμή. Ξεκινώντας με την μέτρηση ογκομετρικής παροχής, υπολογίστηκε πόσο χρόνο χρειάστηκαν $0.1 m^3$ αέρα να περάσουν από την πειραματική διάταξη. Χρησιμοποιώντας τη \prettyref{eq:vol}, υπολογίστηκε

\begin{align*}
\dot{\volume} = \frac{0.1}{102.1} \approx \qty{9.4332d-4}{\metre\cubed\per\second} 
\end{align*}

\noindent Για τοπικές ταχύτητες ρευστού, σε αγωγό κυλινδρικής διατομής, η απλούστερη μέθοδος υπολογισμού ροής είναι ο διαχωρισμός της υπό εξέταση κυκλικής διατομής σε μικρότερες, ίσες περιοχές (i = 1, 2, 3). Οι μετρήσεις λαμβάνονται στο μισό της ακτίνας κάθε δακτυλίου (j =1, 2, 3), και κάθε ταχύτητα είναι αντιπροσωπευτική αυτής της δακτυλοειδούς επιφάνειας. Η διάμετρος της διατομής ήταν περίπου 70 mm, ενώ η θερμοκρασία περιβάλλοντος $26.8 ^oC$. 
Το θερμό νήμα τοποθετήθηκε στα μέσο του πάχους κάθε δακτυλίου (σύμβολο x) των ισεμβαδικά κατανεμημένων προσαυξήσεων, όπως φαίνεται στο \prettyref{fig:flowcalc}

\begin{figure}[!htbp]
\centering    
\subimport{Appendix1/Figs/pdftex}{flowcalculation.pdf_tex}
\caption{Διακριτά σημεία μετρήσεων στην έξοδο πειραματική διάταξης}
\label{fig:flowcalc}
\end{figure}

Ελήφθησαν λοιπόν οι σημειακές μετρήσεις ταχύτητας $U_{ij}(r/r_f)$ για $i = 1, 2, 3$ και $j = 1, 2, 3$, όπως έχουν καταγραφεί στον \prettyref{tab:flowcalc},

\begin{table}[!htbp]
\caption{Μετρήσεις τοπικής ταχύτητας στο επίπεδο $xy$ κυκλικής διατομής}
\centering
\label{tab:flowcalc}
%\setlength\tabcolsep{0pt}
\begin{tabular}{@{}lcccc@{}}
\toprule
& & \multicolumn{3}{c}{$U_{ij} (m/s)$} \\
\cmidrule{3-5}
 Ακτινική Θέση, $i$ & $r/r_f$ & Τμήμα 1 $(j = 1)$ & Τμήμα 2 $(j = 2)$  & Τμήμα 3 $(j = 3)$ \\ 
\midrule
1 & 0.2884 & 0.2481 & 0.2982 & 0.1686 \\

2 & 0.6965 & 0.3331 & 0.3180 & 0.2678 \\

3 & 0.9082 & 0.3355 & 0.3250 & 0.2987 \\
\bottomrule
\end{tabular}
\end{table}

Ο μέσος ρυθμός ροής υπολογίζεται κατά μήκος κάθε διατομής που διασχίζεται χρησιμοποιώντας την εξίσωση 

\begin{align}\label{eq:flow}
\dot{\volume} = \iint_A \,u \,dA
\end{align}

\noindent Ο ρυθμός ροής είναι ίσος με τον μέσο των αντιπροσωπευτικών ροών κάθε δακτυλίου και με διακριτοποίηση των δεδομένων ταχύτητας, η \prettyref{eq:flow} μπορεί να γραφτεί και ως

\begin{align*}
\dot{\volume}_j = 2\pi \, \int_{0}^{r_f} Urdr \, \approx \, 2\pi \, \sum_{i = 1}^{3} \, U_{ij}r_{i} \Delta r_{i}
\end{align*}

\noindent όπου $\Delta r$ είναι η ακτινική απόσταση που χωρίζει κάθε δακτύλιο. Και δεδομένου ότι οι δακτύλιοι που απαρτίζουν την διατομή είναι ίδιου εμβαδού, η ίδια σχέση δύναται να απλοποιηθεί περαιτέρω

\begin{align*}
\dot{\volume}_j &= \, 2\pi \, \sum_{i = 1}^{3} \, U_{ij}r_{i} \Delta r_{i} \\
&= 2\pi \, \left(U_{1j}r_{i} \Delta r_{1} \, + \, U_{2j}r_{2} \Delta r_{2} \, + \, U_{3j}r_{3} \Delta r_{3}\right) \\
&= U_{1j} \, 2\pi r_{i} \Delta r_{1} \, + \, U_{2j} \, 2\pi r_{2} \Delta r_{2} \, + \, U_{3j} \, 2\pi r_{3} \Delta r_{3} &\text{όπου} \quad 2\pi r_{i} \Delta r_{i} = \frac{A}{3}, \, i \in \{1, 2, 3\} \\
&= U_{1j} \, \frac{A}{3} \, + U_{2j} \, \frac{A}{3} \, + \, U_{3j} \, \frac{A}{3}\\
&= \frac{A}{3} \left(U_{1j} \, + U_{2j}  \, + \, U_{3j}\right) \\
&= \frac{A}{3} \, \sum_{i = 1}^{3} \, U_{ij}
\end{align*}

\noindent οπότε, ο μέσος ρυθμός ροής κατά μήκος κάθε τμήματος διέλευσης είναι

\begin{align*}
\dot{\volume}_1 &\approx \qty{0.0009}{\metre\cubed\per\second} & \dot{\volume}_2 &\approx \qty{0.0012}{\metre\cubed\per\second} & \dot{\volume}_3 &\approx \qty{0.0012}{\metre\cubed\per\second}
\end{align*}

\noindent και τελικά, ο μέσος ρυθμός ροής της κυκλικής διατομής, $\overline{\dot{\volume}}$, είναι ο μέσος όρος των επιμέρους ρυθμών ροής

\begin{align*}
\overline{\dot{\volume}} = \frac{1}{3} \, \sum_{j = 1}^{3} \dot{\volume}_j \approx \, \qty{9.4332d-4}{\metre\cubed\per\second}
\end{align*}

\noindent Έχοντας και τις δύο τιμές παροχής, εφαρμόζεται η \prettyref{eq:voler} για να υπολογιστεί το ποσοστιαίο σφάλμα της μέτρησης

\begin{align*}
\dot{\volume}_{er} = 100 \, \frac{9.8224 \times 10^{-4} - 9.4332 \times 10^{-4}}{9.4332 \times 10^{-4}} = 4.11 \%
\end{align*}

\noindent Λαμβάνοντας 7 ακόμη ζεύγη μετρήσεων, εφαρμόζεται η \prettyref{eq:rms} για να εκτιμηθεί το ποσοστιαίο σφάλμα του μετρητή παροχής,

\begin{align*}
\text{err}_{231 - 1250} = \left[\frac{1}{8}\sum_{n = 1}^{8}\frac{\left(\dot{\volume}^{\text{μ.π.}}_{n} - \dot{\volume}^{\text{θ.ν.}}_{n}\right)^2}{\left(\dot{\volume}^{\text{θ.ν.}}_{n}\right)^2}\right]^{1/2} = \, 12.13\%
\end{align*}

\section*{Αποτελέσματα βαθμονόμησης και συμπεράσματα}

\noindent Το \prettyref{plt:flocall1} και το \prettyref{plt:flocall2} προκύπτουν από την επεξεργασία όλων των δεδομένων, όπου η τεταγμένη συμβολίζει το σχετικό σφάλμα μεταξύ των δύο μετρητικών, θεωρώντας, κατά σύμβαση, ως πραγματική τιμή αυτή του θερμού νήματος. Τα δεδομένα αφορούν οχτώ διαφορετικές παροχές αέρα, σταδιακά αυξανόμενες, οι οποίες καθορίστηκαν μέσω πηγής σταθερής τάσης που κινούσε το ανεμιστηράκι αναρρόφησης αέρα της διάταξης.

Παρατηρείται ότι οι μετρήσεις παροχής του μετρητή θετικής μετατόπισης είναι αρκετά κοντά με αυτές του θερμού νήματος, ειδικά από Reynolds 400 και άνω όπου δεν υπερβαίνουν τις 2 ποσοστιαίες μονάδες. Και στα δύο γραφήματα, ωστόσο, η ακρίβεια του μετρητή παροχής φαίνεται να έχει αρκετά μεγάλη απόκλιση στις αρχικές τιμές παροχής (μέχρι και -29,6\%). Αυτό συνάδει με την προβλεπόμενη απόδοση ενός ογκομετριτή παροχής θετικής μετατόπισης (βλ. \prettyref{fig:flowmet} \textit{(β)}). Η παροχή πρέπει να αποκτήσει μια ελάχιστη τιμή ούτως ώστε οι στροφείς, καθώς και ο αναλογικός δείκτης, να υπερβούν τις όποιες δυνάμεις τριβής.

Ακολούθως, παρουσιάζονται τα αποτελέσματα από τη \prettyref{eq:rms},

\begin{align*}
1^{\eta}\, \text{παρτίδα μετήσεων}: err_{231 - 1250} &= 12.13\%, & err_{420 - 1250} &= 3.24\% \\
2^{\eta}\, \text{παρτίδα μετήσεων}: err_{203 - 1196} &= 11.90\%, & err_{401 - 1196} &= 2.13\% \\
\end{align*}

\noindent στη θέση του εύρους έχουν τοποθετηθεί οι αντίστοιχοι Reynolds, των υπό εξέταση τιμών παροχών, για να αποφευχθεί υπερφόρτωση της παρουσίασης.

Η ποσοστιαία απόκλιση μεταξύ του μετρητή θετικής μετατόπισης και του θερμού νήματος, λαμβάνοντας υπόψη όλο το εύρος παροχής μετρήσεων, δεν ξεπερνά το 12.13\%, σε καμία από τις δύο παρτίδες μετρήσεων. Ενώ, αν περιοριστούμε στα εύρη Reynolds 420 με 1250 και 401 με 1196, η διαφορά κυμαίνεται κάτω του 3.24\%.

Δεδομένου του μικρού σφάλματος του μετρητικού από Reynolds 400 και άνω, επιβεβαιώνεται η δυνατότητα υπολογισμού ογκομετρικής παροχής, με χρήση του μετρητή θετικής μετατόπισης, για το εύρος λειτουργίας 1100 με 2000 Reynolds.

\begin{figure}[!htbp]
\centering
\subimport{Appendix1/Figs/pdftex}{flowcallibration1.pdf_tex}
\caption{Απόκλιση μετρήσεων ογκομετρικής παροχής συναρτήσει αυτών θερμικού ανεμόμετρο (πρώτη παρτίδα μετρήσεων)}
\label{plt:flocall1}
\end{figure}

\begin{figure}[H]
\centering
\subimport{Appendix1/Figs/pdftex}{flowcallibration2.pdf_tex}
\caption{Απόκλιση μετρήσεων ογκομετρικής παροχής συναρτήσει αυτών θερμικού ανεμόμετρο (δεύτερη παρτίδα μετρήσεων)}
\label{plt:flocall2}
\end{figure}

\printbibheading
\begin{english}
\printbibliography[heading=subbibliography, type = book, title = {Βιβλία}]
\printbibliography[heading=subbibliography, type = inbook, title = {Κεφάλαια βιβλίων}]
\printbibliography[heading = subbibliography, type = article, title = {Δημοσιεύσεις}]
\end{english}

\end{refsection}