%!TEX root = ../thesis.tex
%*******************************************************************************
%*********************************** First Chapter *****************************
%*******************************************************************************

\chapter{Εισαγωγή}  %Title of the First Chapter

\ifpdf
    \graphicspath{{Chapter1/Figs/Raster/}{Chapter1/Figs/PDF/}{Chapter1/Figs/}}
\else
    \graphicspath{{Chapter1/Figs/Vector/}{Chapter1/Figs/}}
\fi

% ******************************* Nomenclature ****************************************


\begin{chapquote}{Lewis Carroll, \textit{Alice in Wonderland}}
“Begin at the beginning,'' the King said, gravely, "and go on till you
come to an end; then stop.”
\end{chapquote}


%********************************** %First Section  **************************************

\noindent Σήμερα, περισσότερο από ποτέ, η αποτελεσματική αξιοποίηση της ενέργειας αποτελεί προς εξέταση θέμα για ένα ευρύ φάσμα επιστημών. H εξοικονόμηση της ενέργειας, και δει της θερμότητας, μπορεί όχι μόνο να συμβάλλει σημαντικά στην ανάπτυξη και τη βιωσιμότητα των βιομηχανιών, αλλά και στη μείωση λειτουργικών δαπανών. Ο οικονομικός σχεδιασμός καθώς επίσης και η λειτουργία βιομηχανικών μονάδων καθορίζονται από την αποτελεσματική χρήση της θερμότητας εξάλλου.

Η πλειονότητα των εναλλακτών θερμότητας που χρησιμοποιούνται σε χημικές, πετροχημικές και βιοϊατρικές εγκαταστάσεις θερμαίνουν/ψύχουν ρευστά υψηλού ιξώδους. Αυτές οι δραστηριότητες αποτελούν επενδύσεις πολλών εκατομμυρίων ετησίως - τόσο για το κόστος λειτουργίας όσο και τη συντήρησή τους.

Εν τω μεταξύ, τα ιξώδη ρευστά δημιουργούν στρωτές ροές με χαμηλούς αριθμούς Reynolds με αποτέλεσμα χαμηλούς συντελεστές συναγωγής. Συνεπώς, οι χαμηλοί συντελεστές συναγωγής (και μεταφοράς εν γένει) είναι ένα από τα προβλήματα που πρέπει να αντιμετωπιστούν στην βελτιστοποίηση των εναλλακτών στη βιομηχανία.

Τις τελευταίες δεκαετίες έχουν αναπτυχθεί μια πληθώρα μεθόδων που αποσκοπούν στην ενίσχυση των φαινομένων μεταφοράς. Μια εκτεταμένη ανασκόπηση των μεθόδων αυτών γίνεται από τους \citeauthor{1991_Bergles_BOOK_CHAPTER} \cite{1991_Bergles_BOOK_CHAPTER} και \citeauthor{2005_Webb_BOOK} \cite{2005_Webb_BOOK}, οι οποίοι τις διακρίνουν σε ενεργητικές και παθητικές.

Οι ενεργητικές τεχνικές βελτίωσης απαιτούν εξωγενή χρήση ισχύος για τη βελτίωση μεταφοράς θερμότητας. Οι παθητικές τεχνικές, εν αντιθέσει, δεν απαιτούν εξωτερική χρήση ενέργειας, εκτός από αυτήν της αντλίας ή του φυσητήρα για τη μετακίνηση του ρευστού, και περιλαμβάνουν τη χρήση τραχειών επιφανειών, συσκευών στροβιλώδους ροής κλπ.

Κατά συνέπεια, οι παθητικές τεχνικές έχουν περισσότερη απήχηση. Οι δύο πιο κοινές μέθοδοι είναι η εισαγωγή ρευστού διαμέσου καθορισμένης γωνίας οδηγών πτερυγίων, είτε αεροδυναμικά μέσω εφαπτομενικών ρευμάτων αέρα, και έχει διαπιστωθεί ότι αμφότερες είναι εξαιρετικά αποτελεσματικές. Σημαντική βελτίωση μεταφοράς μπορεί να επιτευχθεί, ιδιαίτερα σε στρωτές ροές αγωγών, οι οποίες όταν αναπτυχθούν πλήρως, η μετάδοση θερμότητας παραμένει σταθερή\footnote{οι τοπικοί αριθμοί Nusselt ανεξαρτητοποιούνται των αριθμών Reynolds και Prandtl, λαμβάνοντας τιμές 4.36 και 3.66 για οριακές συνθήκες σταθερής ροής θερμότητας και σταθερής θερμοκρασίας αντίστοιχα \parencites{2011_Bergman_BOOK}{2011_Lienhard_BOOK}}.

Το θέμα αυτό αποτελεί και το θέμα αυτής της πτυχιακής εργασίας. Συγκεκριμένα την παραγωγή περιδινούμενης ροής, η οποία φθίνει ελεύθερα κατά μήκος της διαδρομής του ρευστού, με σκοπό την ενίσχυση της μεταφοράς θερμότητας. Η περιδίνηση δημιουργείται μέσω εφαπτομενικών εισόδων ροής (ή βρόγχους) σε διάταξη δακτυλίου (ή ομόκεντρων κυλίνδρων).

Η παρούσα πτυχιακή εργασία ακολουθεί την εξής δομή: το \textbf{\prettyref{ch:swirlflows}} κάνει μια γενική εισαγωγή στις ροές στροβιλισμού, συγκεκριμένα στις ροές εκείνες όπου ο στροβιλισμός εξασθενεί κατά μήκος της ροής ενός αγωγού (swirl decaying flows), και τη μετάδοση θερμότητας που δύναται αυτές να πετύχουν. Γίνεται επίσης, εν μέρει, βιβλιογραφική ανασκόπηση σε παρόμοιες πειραματικές μελέτες και τις εμπειρικές συσχετίσεις που έχουν αυτές εξάγει. Το \textbf{\prettyref{ch:experiment}} παρουσιάζει την πειραματική διάταξη που κατασκευάστηκε προς διερεύνηση της περιδινούμενης ροής, καθώς επίσης και όλα τα περιφερειακά όργανα μετρήσεων και το σύστημα ανάκτησης δεδομένων. Επίσης συζητιέται η μεθοδολογία εκτέλεσης των πειραμάτων. Το \textbf{\prettyref{ch:uncertaintyanalysis}} δίνει μια λεπτομερή ανάλυση των υπολογισμών που χρησιμοποιήθηκαν, συζητιέται η ανάλυση αβεβαιότητας των μετρήσεων, προσδιορίζεται το σφάλμα των μετρητικών οργάνων και παρουσιάζεται η αυτοματοποιημένη διαδικασία (βλ. \prettyref{code:uncertaintyprop}) μετάδοσής τους στους επακόλουθους υπολογισμούς. Το \textbf{\prettyref{ch:results}} παρουσιάζει τα αποτελέσματα των αναλυτικών υπολογισμών, σχολιάζει τη βελτίωση στη μετάδοση θερμότητας που παρουσιάζουν οι περιδινούμενες ροές και αποσαφηνίζει το κατά πόσον η χρησιμοποιηθείσα ανάλυση ήταν ορθή βάσει των σφαλμάτων στα τελικά αποτελέσματα. Τέλος, το \textbf{\prettyref{ch:suggestions}} κλείνει την πτυχιακή σχολιάζοντας τα τελικά συμπεράσματα που εξήχθησαν και προτείνοντας μελλοντικά βήματα για περαιτέρω διερεύνηση. Ακολουθούν η Βιβλιογραφία και πέντε παραρτήματα τα οποία συμπληρώνουν το βασικό σκέλος της πτυχιακής.