% ************************** Thesis Abstract *****************************
% Use `abstract' as an option in the document class to print only the titlepage and the abstract in greek.

%new command for greek 
\newcommand*{\keywordsgr}[1]{\vspace{1cm}\noindent\textbf{Λέξεις Κλειδιά:} #1}

\begin{abstract-greek}

\noindent Η παρούσα πτυχιακή εργασία ασχολείται με την ενίσχυση μετάδοσης θερμότητας κάνοντας χρήση περιδίνησης, η οποία φθίνει κατά μήκος της ροής, σε διάταξη δακτυλιοειδούς διατομής. Η περιδίνηση δημιουργείτο από κατάλληλα διαμορφωμένες βάσεις, τοποθετημένες ανάντη της ροής, οι οποίες έφεραν συμμετρικά τέσσερις βρόγχους στην περιφέρειά τους, με εργαζόμενο μέσο αέρα. Τα πειράματα έλαβαν χώρα για εύρη αριθμού Reynolds 1100 με 2000, για διάφορες συνθήκες περιδίνησης, μεταβάλλοντας τον αριθμό και την γωνία βρόγχων. Στο πλαίσιο της ποσοτικοποίησης της μετάδοσης θερμότητας, ιδιαίτερη έμφαση δόθηκε στον ενδελεχή προσδιορισμό αλλά και διάδοση σχετικών σφαλμάτων. Τα αποτελέσματα που προέκυψαν ήταν ενθαρρυντικά όσον αφορά την ομοιόμορφη ψύξη και τη μετάδοση θερμότητας - σε ορισμένες περιπτώσεις, η συνολική ενίσχυση μετάδοσης θερμότητας έφτανε το \qty{24}{\percent}. Ωστόσο, οι διατάξεις περιδίνησης, ήταν αισθητά πιο ενεργοβόρες συγκρινόμενες με τις αντίστοιχες αξονικής ροής. Μολονότι τα σφάλματα των υπολογισθέντων μεγεθών βρίσκονταν σε ανεκτά όρια, στα τελικά αποτελέσματα, οι αβεβαιότητες ήταν ιδιαίτερα υψηλές - της τάξεως του \qty{300}{\percent} - καθιστώντας την όποια αξιολόγηση, και πόσο μάλλον σύγκρισή τους, αναξιόπιστη.

\keywordsgr{ροή περιδίνησης; ενίσχυση μετάδοσης θερμότητας; ανάλυση αβεβαιότητας}

\end{abstract-greek}
