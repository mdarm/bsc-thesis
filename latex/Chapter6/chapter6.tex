%!TEX root = ../thesis.tex
%*******************************************************************************
%****************************** Sixth Chapter **********************************
%*******************************************************************************
\chapter{Συμπεράσματα και περαιτέρω έρευνα}\label{ch:suggestions}

\begin{chapquote}{Tom Curran, \textit{Air Force Research Laboratory}}
“The main thing is to keep the main thing the main thing.”
\end{chapquote}

% **************************** Define Graphics Path **************************
\ifpdf
    \graphicspath{{Chapter6/Figs/Raster/}{Chapter6/Figs/PDF/}{Chapter5/Figs/}}
\else
    \graphicspath{{Chapter6/Figs/Vector/}{Chapter6/Figs/}}
\fi


% ******************************* Nomenclature ****************************************

\nomenclature[z-CFD]{CFD}{Computational Fluid Dynamics}
\nomenclature[z-PIV]{PIV}{Particle Image Velocimetry}
\nomenclature[z-AI]{AI}{Artificial Intelligence}
\nomenclature[z-ML]{ML}{Machine Learning}

\noindent Η παρούσα πτυχιακή εργασία επικεντρώθηκε στην τεχνική έγχυσης ρευστού μέσω βρόγχων με σκοπό τη δημιουργία περιδούμενης ροής, και την ενίσχυση της μεταφοράς θερμότητας, κάνοντας χρήση πειραματικών μεθόδων. Η εργασία κινήθηκε σε δύο άξονες: (i) ανάλυση αβεβαιότητας των βασικών υπολογισθέντων θερμοδυναμικών μεγεθών και (ii) την περαιτέρω επεξεργασία των μεγεθών αυτών για τον προσδιορισμό της επίδρασης του αριθμού $\left(\alpha\right)$ και κλίσης $\left(\phi\right)$ βρόγχων στη μετάδοση θερμότητας.

Σκοπός ήταν να δειχτεί ότι οι περιδινούμενες ροές μπορούν να βελτιώσουν τη μετάδοση θερμότητας. Τα πειράματα έλαβαν χώρα σε αγωγό δακτυλιοειδούς διατομής σε συνθήκες στρωτής ροής, για εύρη αριθμού Reynolds 1100 με 2000. Κατά την μελέτη διερευνήθηκαν (i) η θερμοκρασιακή ομοιογένεια, (ii) η βελτίωση μετάδοσης θερμότητας, (iii) η βελτίωση καταναλισκόμενης ισχύος και (iv) η βελτίωση ωφέλιμου δυναμικού. Ιδιαίτερη έμφαση δόθηκε στην επιμελή διάδοση σφαλμάτων, για όλο το φάσμα των υπολογισμών.

Τα αποτελέσματα που προέκυψαν είναι ενθαρρυντικά όσον αφορά την ομοιόμορφη ψύξη και τη μετάδοση θερμότητας, και μπορούν να αποτελέσουν βάση για μελλοντική διερεύνηση της συγκεκριμένης μεθόδου έγχυσης ρευστού. Συγκεκριμένα, διαπιστώθηκε ότι οι περιδινούμενες ροές μπορούν να ενισχύσουν τη μετάδοση θερμότητας μέχρι και \qty{24.23}{\percent}. Παρατηρήθηκε ότι η κλίμακα ενίσχυσης έχει άμεση σχέση με τον αριθμό και την κλίση των βρόγχων. Οι διατάξεις με άρτιο αριθμό βρόγχων παρουσίασαν πιο ομοιογενή ψύξη συγκριτικά αυτών με περιττό αριθμό βρόγχων. Επίσης, για διατάξει με βρόγχους κλίσης 90 μοιρών, παρουσιάστηκε μείωση της μετάδοση θερμότητας που έφτανε μέχρι και \qty{24.45}{\percent}.

Όσον αφορά την καταναλισκόμενη ισχύ και το ωφέλιμο δυναμικό, δεν παρουσιάστηκε βελτίωση σε καμία από τις διατάξεις. Αυτό ήταν αναμενόμενο λαμβάνοντας υπόψη τη φύση των περιδινούμενων ροών. Η μέγιστη απαιτούμενη ισχύς ήταν \qty{1457.43}{\percent} περισσότερη της αξονικής, ενώ η χαμηλότερη τιμή ωφέλιμου δυναμικού έφτασε το \qty{88.93}{\percent} της αξονικής.

Στην ανάλυση αβεβαιότητας, τα σφάλματα στα αποτελέσματα των αρχικών υπολογισμών βρίσκονταν σε ανεκτά όρια, μικρότερα του \qty{4.5}{\percent} των ενδείξεων. Ωστόσο, οι τοπικοί Nusselt παρουσίασαν αβεβαιότητες που κυμάνθηκαν από \qty{5}{\percent} μέχρι \qty{8}{\percent} της ένδειξής τους.

Στον υπολογισμό των δεικτών βελτίωσης, οι αβεβαιότητες φάνηκαν να \enquote{ξεφεύγουν}, κατακρημνίζοντας έτσι την όποια υπόσταση είχαν τα τελικά αποτελέσματα. Συγκεκριμένα, τα σφάλματα ενδείξεων θερμικής βελτίωσης κυμάνθηκαν από \qty{13,16}{\percent} μέχρι \qty{101,35}{\percent} της ένδειξης, αυτά της βελτίωσης κατανάλωσης ισχύος από \qty{58,74}{\percent} μέχρι \qty{1180,64}{\percent} (ναι ξέρω, είναι πολύ!) της ένδειξης και αυτά του ωφέλιμου δυναμικού από \qty{2,9}{\percent} μέχρι \qty{59,64}{\percent} της ένδειξης.

Αυτό είναι ενδεικτικό των περιορισμών (ή των ελλιπών υποθέσεων) που έχει η χρησιμοποιηθείσα ανάλυση και καθίσταται εμφανές ότι η μεθοδολογία οφείλει να επαναξιολογηθεί ως προς την καταλληλότητά της. Το μέγεθος των σφαλμάτων αυτών οφείλεται, κατά κύριο λόγο, στις αβεβαιότητες των παραμέτρων του εκάστοτε μοντέλου παλινδρόμησης, $\sigma _a$ και $\sigma _b$, οι οποίες είναι με τη σειρά τους απόρροια των αβεβαιοτήτων των υπολογισθέντων μεγεθών. Επίσης, όπως και αναφέρθηκε στην υποενότητα~\ref{uncertpre}, οι μεγάλες τιμές αβεβαιοτήτων υποβαθμίζουν την εγκυρότητα της αριθμητικής μεθόδου υπολογισμού των μερικών παραγώγων, που χρησιμοποιήθηκε στη διάδοση των σφαλμάτων (βλ. \prettyref{code:uncertaintyprop}). 

Ο συγγραφέας προτείνει την πιο ενδελεχή μοντελοποίηση σχέσεων συσχέτισης, με χρήση κατάλληλου στατιστικού προγράμματος, καθώς επίσης και μία πιο στιβαρή διαδικασία εύρεσης σφαλμάτων και ακριβέστερης αξιολόγησης διάδοσής τους. Κάτι τέτοιο θα περιλάμβανε την εφαρμογή βαθμονόμησης για κάθε μετρητικό όργανο, δηλαδή τη βαθμονόμηση βάσει της μεγαλύτερης δυνατής μονάδας του συνολικού συστήματος μέτρησης (end-to-end system calibration) \cite{2018_HughW.Coleman_BOOK}. Επίσης η ανάλυση αβεβαιότητας θα μπορούσε να εφαρμοστεί πριν και κατά τη διάρκεια των πειραματικών διεργασιών, για μια πιο ολοκληρωμένη εικόνα των σφαλμάτων $1^{\text{ης}}$ και $Ν^{\text{ης}}$ τάξεως ξεχωριστά όπως περιγράφουν οι \citeauthor{2018_HughW.Coleman_BOOK} \cite{2018_HughW.Coleman_BOOK} και οι \citeauthor{2021_Moffat_BOOK} \cite{2021_Moffat_BOOK}.

Για την περαιτέρω πειραματική διερεύνηση της ενίσχυσης μετάδοσης θερμότητας των διατάξεων βρόγχων, προτείνεται χρήση PIV (Particle Image Velocimetry) ούτως ώστε να υπάρχουν πειραματικά δεδομένα που αφορούν το πεδίο ταχύτητας. Τούτου λεχθέντος, η πειραματική διάταξη θα μπορούσε να εξοπλιστεί με θερμοστοιχεία που να λαμβάνουν τοπικές θερμοκρασίες στην εσωτερική επιφάνεια του εξωτερικού κυλίνδρου, ούτως ώστε να υπάρχει μια πιο ολοκληρωμένη εικόνα για τη θερμοδυναμική συμπεριφορά του συστήματος.

Δυστυχώς οι συσχετίσεις $\overline{Nu} = f(Re)$ δεν μπόρεσαν να συγκριθούν με αντίστοιχες της βιβλιογραφίας για τον πολύ απλό λόγο ότι παρεμφερείς πειραματικές εργασίες διαπραγματεύονται πεδία τυρβώδων ροών. Ως εκ τούτου, προτείνεται η επέκταση της πειραματικής διερεύνησης των διατάξεων βρόγχων σε τυρβώδη πεδία ροής, τα οποία παρουσιάζουν αρκετά διαφορετικό υδροδυναμικό χαρακτήρα με τα συγκρινόμενα στρωτά \parencites{2016_Chen}{2016_Chena}{2018_Chen}.

Ένα άλλο ζητούμενο της μελλοντικής έρευνας είναι η χρήση των μεθόδων Υπολογιστικής Ρευστοδυναμικής (Computational Fluid Dynamics) στη βελτιστοποίηση της μετάδοσης θερμότητας για τις διατάξεις βρόγχων. Μια τέτοια υπολογιστική διαδικασία όχι μόνο θα έδινε εκτιμήσεις για την περιγραφή του πεδίου ροής, αλλά και θα διευκόλυνε, κατά κόρον, στην βελτιστοποίηση των διατάξεων βρόγχων, με την ανάπτυξη πολλαπλών παραμετρικών μελετών και ανάλυσης ευαισθησίας των σημαντικότερων παραμέτρων που διέπουν την εκάστοτε διάταξη. Επίσης, θα μπορούσαν να δοκιμαστούν διάφορες διατάξεις, για πολλές συνθήκες, σε σχετικά λιγότερο χρόνο με μόνες επιβαρύνσεις αυτές της υπολογιστικής ισχύος και του υπολογιστικού χρόνου.

Κλείνοντας, δεδομένης της ύπαρξης πειραματικών και υπολογιστικών τιμών, προτείνεται η χρήση τεχνικών ανάλυσης δεδομένων μέσω μεθόδων Τεχνητής νοημοσύνης (AI) και Μηχανικής εκμάθησης (ML). Αν και σχετικά πρόσφατος κλάδος της ρευστοδυναμικής \parencites{2022_Zhang}{2022_Brunton}, φαίνεται να παντρεύει τις παραδοσιακές μεθόδους επίλυσης, ενσωματώνοντας πειραματικά και υπολογιστικά δεδομένα, και παρέχοντας κατ' αυτόν τον τρόπο μια διαφορετική οπτική στην βελτιστοποίηση και την παραμετρική ανάλυση στην οποία έκανε νύξη η συγκεκριμένη εργασία.