%!TEX root = ../thesis.tex
%*******************************************************************************
%****************************** Third Chapter **********************************
%*******************************************************************************
\chapter{Εγκατάσταση δοκιμών}\label{ch:experiment}

\begin{chapquote}{Richard Feynman\footnote{Αυτό έχει ειπωθεί στις διαλέξεις του Feynman στο Cornell, το 1964. Κομμάτι των διαλέξεων αυτών είναι διαθέσιμο στο σύνδεσμο \href{https://youtu.be/OL6-x0modwY}{https://youtu.be/OL6-x0modwY}. Ένα πιο πλήρες κείμενο μπορεί να βρεθεί στον ιστότοπο του \citeauthor{Pomeroy2012} \cite{Pomeroy2012}.}}
“It doesn’t matter how beautiful your theory is, it doesn’t matter how smart you are.
If it doesn’t agree with experiment, it’s wrong.”
\end{chapquote}

% **************************** Define Graphics Path **************************
\ifpdf
    \graphicspath{{Chapter3/Figs/Raster/}{Chapter3/Figs/PDF/}{Chapter3/Figs/}}
\else
    \graphicspath{{Chapter3/Figs/Vector/}{Chapter3/Figs/}}
\fi


% ******************************* Nomenclature ****************************************

\nomenclature[a-$z/L$]{$z/L$}{ανηγμένο μήκος αντίστασης (εσωτερικού κυλίνδρου)}

\noindent Στο κεφάλαιο αυτό περιγράφεται η πειραματική διάταξη που κατασκευάστηκε για την διερεύνηση της ενίσχυσης μετάδοσης θερμότητας περιδινούμενων ροών. Περιγράφονται επίσης οι πειραματικές συσκευές και τα χρησιμοποιηθέντα μετρητικά όργανα, και γίνεται αναφορά στην πειραματική διαδικασία λήψης μετρήσεων.

\section{Πειραματική διάταξη}

\noindent Σκαρίφημα της πειραματικής διάταξης απεικονίζεται στο \prettyref{fig:schematic}. Πρόκειται για σύστημα που απαρτίζεται από (i) τη βάση ροής, (ii) τη διάταξη ομόκεντρων κυλίνδρων και (iii) από έναν ανεμιστήρα αναρρόφησης. Η σχεδίαση των ομόκεντρων κυλίνδρων και βάσεων βρόγχων ήταν τέτοια ώστε ο λόγος πάχους του δακτυλίου των πρώτων $\left(R_{\text{εξ.}} - R_{\text{εσ.}}\right)$ με τη διάμετρο των βρόγχων $\left(D_{\beta}\right)$ να είναι 1:1. Έτσι εξασφαλίζονται συνθήκες ροής "pure swirl flow" \cite{1991_Legentilhomme}. Οι βάσεις ροής ήταν τοποθετημένες ανάντη της διάταξης ενώ ο ανεμιστήρας αναρρόφησης κατάντη της, και ήταν ικανός να παρέχει ταχύτητες μέχρι και \qty{2}{\metre\per\second} στην έξοδο της διάταξης. 

\clearpage

\begin{figure}[tbp]
\centering
\subimport{Chapter3/Figs/pdftex}{schematic.pdf_tex}
\caption{Σκαρίφημα πειραματικής διάταξης}\label{fig:schematic}
\end{figure}

\subsection{Βάσεις ροής}

\noindent Όπως φαίνεται στο \prettyref{fig:swirlgenerators}, οι βάσεις ροής ήταν πέντε στο σύνολό τους: μία βάση αξονικής και τέσσερις περιδινούμενης ροής. Οι βάσεις περιδινούμενης (θα αναφέρονται στο κείμενο και ως βάσεις βρόγχων) είχαν συμμετρικά τέσσερις βρόγχους στην περιφέρειά τους, με κλίση $\left(\phi\right)$ σε σχέση με την εγκάρσια διατομή τους. Όλες εκτυπώθηκαν με τριδιάστατο εκτυπωτή και είναι κατασκευασμένες από υλικό με αντοχή σε θερμικά φορτία μέχρι \qty{150}{\degreeCelsius}. Σχεδιάστηκαν με τρόπο τέτοιο ώστε να γίνεται καλή εφαρμογή με το σύστημα κυλίνδρων, ενώ στη βάση τους υπάρχει μια μικρή οπή για τη διέλευση των θερμοστοιχείων που είναι τοποθετημένα στον εσωτερικό κύλινδρο. Περιδίνηση ροής δημιουργείται εφαρμόζοντας την κατάλληλη βάση, η οποία εισάγει εφαπτομενικό ρεύμα αέρα και ψύχει τον εσωτερικό κύλινδρο ο οποίος θερμαίνεται.

\clearpage

\begin{figure}[tbp]
\centering
\subimport{Chapter3/Figs/pdftex}{swirlGenerators.pdf_tex}
\caption{Βάσεις ροής}\label{fig:swirlgenerators}
\end{figure}

\subsection{Διάταξη ομόκεντρων κυλίνδρων}

\noindent Το βασικό κομμάτι της εγκατάστασης αποτελείται από δύο ομόκεντρους κυλίνδρους, τον εσωτερικό (ή αντίσταση) και εξωτερικό, στο διάκενο των οποίων ρέει το ρευστό. Η διάταξη των ομόκεντρων κυλίνδρων φαίνεται στο \prettyref{fig:concylinders}.


\begin{figure}[htbp]
\centering
\subimport{Chapter3/Figs/pdftex}{concylinders.pdf_tex}
\caption{Σύστημα ομόκεντρων κυλίνδρων: (α) Εξωτερικός (β) Εσωτερικός}\label{fig:concylinders}
\end{figure}
	
 
\subsubsection{Εξωτερικός κύλινδρος}

\noindent Ως εξωτερικός κύλινδρος χρησιμοποιήθηκε σωλήνας Plexiglass εσωτερικής διαμέτρου \qty{40}{\milli\metre} με πάχος \qty{5}{\milli\metre} και μήκος \qty{895}{\milli\metre}. Η επιλογή του υλικού ήταν τέτοια ώστε να διευκολύνεται η περαιτέρω κατεργασία του, σε περίπτωση που αυτή χρειαζόταν, και οι μηχανικές του ιδιότητες να συνδράμουν στην ομαλή διεξαγωγή των πειραμάτων - το υλικό έχει χαμηλό συντελεστή θερμικής αγωγιμότητας και είναι διαφανές.

Υλικό ανακλώμενης επιφάνειας τυλίχτηκε γύρω από τον εξωτερικό κύλινδρο, ώστε να περιοριστούν οι απώλειες θερμότητας λόγω ακτινοβολίας από τον εσωτερικό προς τον εξωτερικό κύλινδρο. Προστέθηκε επίσης άλλη μια στρώση μονωτικού υλικού προς μείωση απωλειών λόγω αγωγής.

Τέλος, στην κορυφή του εξωτερικού κυλίνδρου, τοποθετήθηκαν τρεις κοχλίες $Μ14$ για να διασφαλιστεί η εκκεντρότητα του με τον εσωτερικό κύλινδρο και να μειωθούν, όσο αυτό ήταν εφικτό, τα χωρικά σφάλματα.

\subsubsection{Εσωτερικός κύλινδρος}

\noindent Για τη δημιουργία του εσωτερικού κυλίνδρου χρησιμοποιήθηκε σωλήνας διαμέτρου \qty{22}{\milli\metre}, πάχους \qty{1}{\milli\metre} και μήκους \qty{900}{\milli\metre}, με καλυμμένη τη μία του έξοδο, ανάντη του συστήματος. Εντός του σωλήνα τοποθετήθηκε κεραμικό υλικό, ανθεκτικό σε θερμικά φορτία, και περί αυτού τυλίχτηκε ομοιόμορφα σύρμα αντίστασης σε κατάλληλα σχηματισμένες αυλακώσεις, για τα επιτευχθεί όσο το δυνατόν πιο ομοιόμορφη θέρμανση. Κατά τη διάρκεια της τοποθέτησης του καλωδίου αντίστασης, δόθηκε ιδιαίτερη προσοχή στην αποφυγή περιτύλιξης για αποφυγή βραχυκύκλωσής του.

Η αντίσταση τροφοδοτείτο από σταθερή πηγή τάσης τύπου MERSAN, με ονομαστική τιμή \qty{50}{\volt}. Επειδή η τάση που διέπει το καλώδιο δεν πρέπει να μεταδοθεί στον χάλκινο σωλήνα, χρησιμοποιήθηκε μονωτική ταινία τύπου Mica η οποία είναι ηλεκτρικά καλός μονωτής και ταυτόχρονα θερμικά αγώγιμο υλικό. Η τοποθέτηση της ταινίας ήταν τέτοια ώστε να επικαλύπτει επακριβώς το διάκενο μεταξύ αυλακώσεων και τυλιγμένου σύρματος, και να βρίσκεται όσο το δυνατόν πλησιέστερα στο χάλκινο σωλήνα.

Για τον προσδιορισμό της θερμοκρασίας κατά μήκος της επιφάνειας του χάλκινου σωλήνα χρησιμοποιήθηκαν θερμοστοιχεία τύπου Κ, τοποθετημένα, με ομοιόμορφο μοτίβο, κατά μήκος της εσωτερικής επιφάνειάς του, με θερμικά αγώγιμη κόλλα. Επειδή τα θερμοστοιχεία τοποθετήθηκαν στο εσωτερικό του σωλήνα, για μην εμποδίσουν το πεδίο ροής, πακτώθηκαν στο κάτω άκρο της αντίστασης με σιλικόνη - θερμικά αγώγιμο υλικό μεν, ηλεκτρικά μονωτικό δε. Στο κάτω μέρος της αντίστασης τοποθετήθηκε κατάλληλη βάση ώστε να προσδένεται με τις βάσεις ροής, και η οποία αποτελείτο από PTFE, το οποίο έχει χαμηλό συντελεστή μετάδοσης θερμότητας και αντοχή σε υψηλά θερμικά φορτία.

\subsection{Ανεμιστήρας αναρρόφησης}

\noindent Στο πάνω μέρος της διάταξης τοποθετήθηκε ανεμιστήρας αναρρόφησης τύπου Corsair iCUE SP120, ονομαστικής τάσης \qty{24}{\volt}. Για τον έλεγχο της ροής, ο ανεμιστήρας συνδεόταν με πηγή συνεχούς ρεύματος EZ GP-4303D DC, ονομαστικής τάσης \qty{40}{\volt}. Για την εφαρμογή του ανεμιστήρα με τη διάταξη ομόκεντρων κυλίνδρων, αλλά και για την σύνδεσή του με τον μετρητή ογκομετρικής παροχής, δημιουργήθηκε ειδική θήκη. Η θήκη αυτή αποτελείτο από το ίδιο υλικό με αυτό των βάσεων ροής, και η ανοχή των διαστάσεών της δεν υπερέβαινε, σύμφωνα με τον κατασκευαστή τουλάχιστον, τα \qty{0.2}{\milli\metre}.

\section{Μετρητικά όργανα και ανάκτηση δεδομένων}

\noindent Για την πειραματική διερεύνηση της ενίσχυσης μετάδοσης θερμότητας στον εσωτερικό κύλινδρο, λήφθηκαν μετρήσεις θερμοκρασίας, χρόνου και τάσεως-έντασης ρεύματος. Οι πρώτες έγιναν με χρήση θερμοστοιχείων τύπου Κ, αυτές του χρόνου με συνδυασμό χρονομέτρου χειρός και μετρητή παροχής, και αυτές των τάσεων-έντασης ρεύματος με πολύμετρο ακριβείας. Οι μετρήσεις των θερμοκρασιών καταγράφονταν σε ηλεκτρονικό υπολογιστή, ενώ οι υπόλοιπες λήφθηκαν χειρωνακτικά. Η ανάλυση της πειραματικής αβεβαιότητας των μετρήσεων καθώς επίσης και των υπολογισθέντων μεγεθών, αναφέρονται στην υποενότητα~\ref{uncertaintyofmeasurements}.

\subsection{Θερμοστοιχεία τύπου Κ}

\noindent Για μετρήσεις θερμοκρασιών, χρησιμοποιήθηκαν θερμοστοιχεία τύπου Κ, τα οποία συγκολήθηκαν στο εργαστήριο λίγο πριν την τοποθέτησή τους στον εσωτερικό κύλινδρο. Στο σύνολό τους ήταν έντεκα: εννέα μετρούσαν τοπικές θερμοκρασίες κατά μήκος της αντίστασης, ένα κατάντη της διάταξης για την θερμοκρασία εισόδου του ρευστού και άλλο ένα ανάντη της διάτξης για μέτρηση της θερμοκρασίας εξόδου.

\begin{figure}[h!]
\centering
\subimport{Chapter3/Figs/pdftex}{thermocouples.pdf_tex}
\caption{Θέσεις θερμοστοιχείων της πειραματικής διάταξης}
\end{figure}

Για τη συνεχή καταγραφή της θερμοκρασίας, επί και εκτός της αντίστασης, χρησιμοποιήθηκε πολύμετρο τύπου KEITHLEY 2700. Δυστυχώς δεν κατέστη δυνατή η συνεχής αποθήκευση των δεδομένων αυτών λόγω \enquote{διαξιφισμού} λογισμικών πολυμέτρου-υπολογιστή. Ωστόσο, χρησιμοποιήθηκε επιτυχώς για τον συνεχή έλεγχο των θερμοκρασιών, μέχρι αυτές να συγκλίνουν προς μία σταθερή τιμή. 

Τα θερμοστοιχεία, εξαιτίας της σχετικά απλής χρήσης τους, αποτελούν την πλέον διαδεδομένη μέθοδο μέτρησης της θερμοκρασίας. Ωστόσο, οι μετρήσεις μπορούν να επηρεαστούν (και συνήθως επηρεάζονται) σημαντικά από απώλειες θερμότητας λόγω συναγωγής και της διαδικασίας συγκόλλησής τους. Οπότε οι μετρήσεις τείνουν να υπερεκτιμούνται αυτών των πραγματικών τιμών λόγω φαινομένων αδράνειας και ενεργειακών απωλειών \cite{1994_FluidDynamicsRhodeSaintGenese_BOOK}. Στην παρούσα εργασία, δεν έγινε καμία προσπάθεια αντιμετώπισης των σφαλμάτων αυτών.

\subsection{Μετρητής φυσικού αερίου}

\noindent Ο μετρητής φυσικού αερίου συνδέεται στην έξοδο του συγκροτήματος ανεμιστήρα. Χρησιμοποιήθηκε σε συνδυασμό με χρονόμετρο, όπου, σε κάθε πείραμα, καταγραφόταν ο απαιτούμενος χρόνος για τη διέλευση \qty{0.1}{\metre\cubed} αέρα. Όλες οι παροχές του αέρα, για κάθε διάταξη ροής, προέρχονται από τον ανεμιστήρα αναρρόφησης, οι στροφές του οποίου ρυθμίζονταν κάνοντας χρήση πηγής συνεχούς ρεύματος.

Λόγω παλαιότητάς του πραγματοποιήθηκε \enquote{σχετική βαθμονόμηση} με θερμό νήμα, ώστε να υπάρξει μια εκτίμηση για την εγκυρότητά του. Περισσότερες πληροφορίες για διαδικασία αναφέρονται στο \prettyref{app:calibration}.

\subsection{Ψηφιακό πολύμετρο}

\noindent Λόγω παλαιότητας των πηγών τάσεως, οι ενδείξεις των εγγενών ενδείξεών τους δεν ήταν αξιόπιστες. Κρίθηκε σκόπιμο λοιπόν να χρησιμοποιηθεί πολύμετρο ακριβείας τύπου DT-9205A, για τη μέτρηση των τιμών τάσης και έντασης ρεύματος του ανεμιστήρα και της αντίστασης.

\section{Διερευνηθείσες ροές και μεθοδολογία μετρήσεων}

\noindent Έχοντας συναρμολογήσει την όλη διάταξη, σε πρώτο στάδιο, ορίστηκαν οι συνθήκες λειτουργίας των πειραμάτων και η μεθοδολογία για την αξιολόγηση μετάδοσης θερμότητας των βάσεων περιδίνησης. Αποφασίστηκε ότι σε κάθε πειραματική διάταξη θα λαμβάνονται δεδομένα για τέσσερις αριθμούς Reynolds: 1100, 1400, 1700 και 2000 - εύρος που εξασφαλίζει συνθήκες στρωτής ροή για την εν λόγο διάταξη \cite{Dou2005}. Όλα τα μεγέθη ήταν χρονικά αμετάβλητα, δηλαδή επικρατούσαν συνθήκες μόνιμης ροής. Για να επιτευχθεί αυτό, υπήρξε συνεχής έλεγχος των ενδείξεων των θερμοστοιχείων μέχρι οι τιμές τους να συγκλίνουν προς μία σταθερή τιμή. Τέλος, για να μπορέσει να καταστεί δυνατή η σύγκριση μετάδοσης θερμότητας μεταξύ των διατάξεων, η ηλεκτρική τροφοδότηση του εσωτερικού κυλίνδρου ήταν η ίδια για όλα τα πειράματα \cite{1974_Bergles}, με ένταση ρεύματος \qty{0,63}{\ampere} και τάσης \qty{29.5}{\volt}, οι οποίες δίνουν την παρεχόμενη ηλεκτρική ισχύ της αντίστασης $\dot{Q}_{res.} = \qty{18.85}{\watt}$.

Έγιναν λοιπόν πειράματα, με όλους τους πιθανούς συνδυασμούς αριθμών $\left(\alpha\right)$ και κλίσης $\left(\phi\right)$ βρόγχων, για τέσσερις αριθμούς Reynolds. Επίσης, για το ίδιο εύρος Reynolds, έλαβαν χώρα άλλα τέσσερα πειράματα χρησιμοποιώντας τη διάταξη αξονικής ροής, ώστε να υπάρχουν τιμές αναφοράς για την ενίσχυση μετάδοσης θερμότητας. Στο σύνολό τους, τα πειράματα ήταν 68.

Τα πειράματα έλαβαν χώρα σε διάστημα δύο μηνών, τον Οκτώβριο και το Νοέμβριο του έτους 2019 . Για λόγους συνέπειας, τα πειράματα ξεκινούσαν τις πρωινές ώρες και ολοκληρώνονταν νωρίς το μεσημέρι της ίδιας ημέρας ώστε να μειωθεί η συνεισφορά του παράγοντα εξωτερικής θερμοκρασίας, η οποία αυξανόταν αισθητά το μεσημέρι. Πριν την εκκίνηση των πειραμάτων της ημέρας, κάθε φορά, παρατηρούνταν με σχολαστικότητα οι ενδείξεις των θερμοστοιχείων κατά μήκος του εσωτερικού κυλίνδρου. Αυτές θα έπρεπε πάντα να έχουν αύξουσα τιμή, διαβάζοντας τες ανάντη του κυλίνδρου, λόγο φυσικής κυκλοφορίας του αέρα.

Ιδιαίτερη προσοχή δόθηκε στην τοποθέτηση των συσκευών λήψης δεδομένων και των μετρητικών συστημάτων, ώστε να αποφευχθούν πιθανές μετακινήσεις και άσκοπες αποσυναρμολογήσεις την ώρα των πειραμάτων. Λάθη που σχετίζονται με την  ελλιπή πρόσδεση των κυλίνδρων, πιθανόν να δημιουργήσουν διαρροές, άρα και διαφορετικές συνθήκες ροής συγκριτικά με αυτές των υπολοίπων. Επιπλέον, τυχόν ανεπάρκεια στην ευθυγράμμιση των εσωτερικού και εξωτερικού κυλίνδρων είναι ικανή να επάγει ανεπιθύμητες αλλαγές στο πεδίο ροής, και να οδηγήσει σε λανθασμένες μετρήσεις. Γενικά δεν επιχειρήθηκε να αξιολογηθεί η αβεβαιότητα που σχετίζεται με την ανεπάρκεια ευθυγράμμισης (χωρικό σφάλμα). Κατά τη διάρκεια των πειραμάτων δεχτήκαμε, κατά σύμβαση, ότι οι ανοχές μεταξύ των προσδεμένων μερών, καθώς και η ευθυγράμμισή τους, ελεγχόταν αρκετά αποτελεσματικά, και ότι τυχόν αβεβαιότητες, που θα μπορούσαν να αποδοθούν σε κακή συνδεσμολογία, ήταν αμελητέες.

\subsubsection{Αποτελέσματα δοκιμών}

\noindent Τα πειραματικά αποτελέσματα καταγράφονται στο \prettyref{app:measurements}. Οι ληφθείσες μετρήσεις αξονικής ροής εμπεριέχονται όλες στον Πίνακα~\ref{axialmeas}. Ενώ οι μετρήσεις περιδινούμενων ροών, λόγω αυξημένου όγκου δεδομένων, εμπεριέχονται στις υπόλοιπες σελίδες. Συγκεκριμένα: οι μετρήσεις θερμοκρασίας βρίσκονται στους Πίνακες~\ref{tempmeas45-60}-\ref{tempmeas75-90}, οι μετρήσεις τάσης και έντασης ρεύματος του ανεμιστήρα στον Πίνακα~\ref{tab:flowmeas} και τέλος, οι μετρήσεις χρόνου για τη διέλευση \qty{0.1}{\metre\cubed} αέρα, στον Πίνακα~\ref{tab:flowmeas}.

Στο επόμενο κεφάλαιο θα παρουσιαστούν η ανάλυση δεδομένων καθώς επίσης και η εκτιμόμενη αβεβαιότητα των μετρητικών οργάνων και των υπολογισθέντων μεγεθών.