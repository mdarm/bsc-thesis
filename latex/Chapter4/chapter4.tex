%!TEX root = ../thesis.tex
%*******************************************************************************
%****************************** Fourth Chapter **********************************
%*******************************************************************************
\chapter{Yπολογισμοί και ανάλυση αβεβαιότητας}\label{ch:uncertaintyanalysis}

\begin{chapquote}{Winston Churchill}
“True genius resides in the capacity for evaluation of uncertain, hazardous, and conflicting
information.”
\end{chapquote}


% **************************** Define Graphics Path **************************
\ifpdf
    \graphicspath{{Chapter4/Figs/Raster/}{Chapter4/Figs/PDF/}{Chapter4/Figs/}}
\else
    \graphicspath{{Chapter4/Figs/Vector/}{Chapter4/Figs/}}
\fi


% ******************************* Nomenclature ****************************************
\nomenclature[a-$Re$]{$Re$}{Αριθμός Reynolds}
\nomenclature[a-$Nu$]{$Nu$}{Αριθμός Nusselt}
\nomenclature[a-$Bi$]{$Bi$}{Αριθμός Biot}
\nomenclature[a-$h$]{$h$}{Συντελεστής θερμικής συναγωγιμότητας, \unit{\watt\per\metre\squared\per\kelvin}}
\nomenclature[a-$\overline{Nu}$]{$\overline{Nu}$}{Μέσος αριθμός Nusselt}
\nomenclature[a-$\overline{Nu}_{\scriptsize{avg}}$]{$\overline{Nu}_{\scriptsize{avg}}$}{Αντιπροσωπευτικός αριθμός Nusselt}
\nomenclature[a-$Pr$]{$Pr$}{Αριθμός Prandtl}
\nomenclature[a-$J$]{$J$}{Ιακωβιανός πίνακας}
\nomenclature[a-\volume]{\volume}{Όγκος, \unit{\metre\cubed}}
\nomenclature[a-$\dot{\volume}$]{$\dot{\volume}$}{Ογκομετρική παροχή, \unit{\metre\cubed\per\second}}
\nomenclature[a-$\dot{Q}_{sys.}$]{$\dot{Q}_{sys.}$}{Ροή θερμότητας προς τον αέρα, \unit{\joule\per\second}}
\nomenclature[a-$\dot{W}$]{$\dot{W}$}{Παρεχόμενη ηλεκτρική ισχύς ανεμιστήρα, \unit{\joule\per\second}}
\nomenclature[a-$\dot{W}_{\scriptsize{avg}}$]{$\dot{W}_{\scriptsize{avg}}$}{Αντιπροσωπευτική ηλεκτρική ισχύς ανεμιστήρα, \unit{\joule\per\second}}
\nomenclature[a-$\dot{Q}_{res.}$]{$\dot{Q}_{res.}$}{Παρεχόμενη ηλεκτρική ισχύς στην αντίσταση, \unit{\joule\per\second}}
\nomenclature[a-$A_{heated}$]{$A_{heated}$}{Εμβαδόν θερμαινόμενης επιφάνειας, \unit{\metre\squared}}
\nomenclature[a-$\dot{q}_{conv.}$]{$\dot{q}_{conv.}$}{Ροή θερμικής ενέργειας στην αντίσταση, \unit{\joule\per\second\per\metre\squared}}
\nomenclature[a-$V_{\text{ανεμ.}}$]{$V_{\text{ανεμ.}}$}{Τάση τροφοδοσίας ανεμιστήρα, \unit{\volt}}
\nomenclature[a-$I_{\text{ανεμ.}}$]{$I_{\text{ανεμ.}}$}{Ένταση ρεύματος τροφοδοσίας ανεμιστήρα, \unit{\ampere}}
\nomenclature[a-$Τ_{mean}$]{$Τ_{mean}$}{Μέση θερμοκρασία διάταξης, \unit{\degreeCelsius}}
\nomenclature[a-$Τ_{\sigma}$]{$Τ_{\sigma}$}{Τυπική απόκλιση θερμοκρασίας, \unit{\degreeCelsius}}
\nomenclature[a-$w_x$]{$w_x$}{Βάρος εκάστοτε δεδομένων}
\nomenclature[a-$s^2$]{$s^2$}{Διασπορά εκάστοτε δεδομένων}
\nomenclature[a-$A_{ann.}$]{$A_{ann.}$}{Εμβαδόν διατομής δακτυλιοειδούς σωλήνα, \unit{\metre\squared}}
\nomenclature[a-$\overline{U}$]{$\overline{U}$}{Μέση ταχύτητα, \unit{\metre\per\second}}
\nomenclature[a-$C_p$]{$C_p$}{Ειδική θερμοχωρητικότητα, \unit{\kilo\joule\per\kilogram\per\kelvin}}
\nomenclature[a-$\dot{m}$]{$\dot{m}$}{Παροχή μάζας, \unit{\kilogram\per\second}}
\nomenclature[a-$D_\beta$]{$D_\beta$}{Διάμετρος οπής βρόγχου, \unit{\metre}}
\nomenclature[a-$D_{\text{εξ.}}$]{$D_{\text{εξ.}}$}{Διάμετρος εξωτερικού κυλίνδρου, \unit{\metre}}
\nomenclature[a-$R_{\text{εξ.}}$]{$R_{\text{εξ.}}$}{Ακτίνα εξωτερικού κυλίνδρου, \unit{\metre}}
\nomenclature[a-$D_{\text{εσ.}}$]{$D_{\text{εσ.}}$}{Διάμετρος εσωτερικού κυλίνδρου, \unit{\metre}}
\nomenclature[a-$R_{\text{εσ.}}$]{$R_{\text{εσ.}}$}{Ακτίνα εσωτερικού κυλίνδρου, \unit{\metre}}
\nomenclature[a-$\left\langle \overline{X} \right\rangle$]{$\left\langle \overline{X} \right\rangle$}{Ομαδοποιημένος μέσος όρος}
\nomenclature[a-$p$]{$p$}{Περίμετρος συστραμμένης ταινίας, m}

\nomenclature[z-RSS]{RSS}{Root-Sum-Squared}
\nomenclature[z-DRE]{DRE}{Data Reduction Equation}

\nomenclature[s-$\chi$]{x}{Θέση θερμοστοιχείων κατά μήκος της αντίστασης, \unit{\metre}}
\nomenclature[s-x, y, z]{x, y, x}{Καρτεσιανές συνιστώσες, \unit{\metre}}
\nomenclature[s-r, θ, z]{r, θ, z}{Κυλινδρικές συνιστώσες}
\nomenclature[s-ρευ.]{ρευ.}{ρευστού}

\nomenclature[g-$\delta$]{$\delta$}{Πάχος οριακού στρώματος, \unit{\metre}}
\nomenclature[g-$\rho$]{$\rho$}{Πυκνότητα, \unit{\kilogram\per\meter\cubed}}
\nomenclature[g-$\kappa$]{$\kappa$}{Θερμική αγωγιμότητα, \unit{\watt\per\metre\per\kelvin}}
\nomenclature[g-$\mu$]{$\mu$}{Δυναμικό ιξώδες, \unit{\kilogram\per\meter\per\second}}
\nomenclature[g-$\left\langle\sigma _X\right\rangle$]{$\left\langle\sigma _X\right\rangle$}{Ομαδοποιημένη τυπική απόκλιση}
\nomenclature[g-$\left\langle\sigma _{\overline{X}}\right\rangle$]{$\left\langle\sigma _{\overline{X}}\right\rangle$}{Ομαδοποιημένη τυπική απόκλιση των μέσων}
\nomenclature[g-$\delta X_{0}$]{$\delta X_{0}$}{Σφάλμα μηδενικής τάξης}
\nomenclature[g-$\delta X_{1}$]{$\delta X_{1}$}{Σφάλμα πρώτης τάξης}
\nomenclature[g-$\delta X_{N}$]{$\delta X_{N}$}{Σφάλμα $N^{\text{ής}}$ τάξης}
\nomenclature[g-$\delta X_{sens.}$]{$\delta X_{sens.}$}{Σφάλμα ευαισθησίας}

\nomenclature[x-$\nabla$]{$\nabla$}{Τελεστής Euler}

\noindent Στο κεφάλαιο αυτό περιγράφεται η αναλυτική μεθοδολογία που αναπτύχθηκε για τη θερμοδυναμική διερεύνηση όλων των διατάξεων ροών ομόκεντρων κυλίνδρων (περιδινούμενων και αξονικών). Συγκεκριμένα: (i) αναφέρεται η ανάλυση δεδομένων καθώς και οι υποθέσεις στις οποίες αυτή βασίστηκε, (ii) παρουσιάζεται λεπτομερώς η ανάλυση αβεβαιότητας των μετρούμενων τιμών, και η συνεισφορά αυτών στα τελικά αποτελέσματα και (iii) παρουσιάζονται οι σχετικές αβεβαιότητες που θα έχουν τα αποτελέσματα των αναλυτικών υπολογισμών για τα εύρη λειτουργίας στα οποία ελήφθησαν οι πειραματικές μετρήσεις.

Ο συγγραφέας οφείλει να ομολογήσει σε αυτό το σημείο ότι ο τρόπος συγγραφής του παρόντος κεφαλαίου συνιστά πλεονασμό. Πράγματι, πειραματικής φύσεως διατριβές \parencites{2015_Μηλιδόνης_DISSERTATION}{2020_Δόγκας_DISSERTATION} είθισται να αναφέρονται στην ανάλυση αβεβαιότητας περιληπτικά, συμπεριλαμβάνοντας επιμέρους επεξηγήσεις στα παραρτήματα. Ωστόσο, επειδή η αρχική πρόθεση του συγγραφέα ήταν να στήσει ένα αρκετά στιβαρό πείραμα, αξιοποιώντας ό,τι εργαλεία είχε στη διάθεση του, και επειδή ο περισσότερος φόρτος εργασίας (υλοποίηση κώδικα, αναλυτικοί υπολογισμοί, ανασκόπηση βιβλιογραφίας κλπ.) καταναλώθηκε για τον σκοπό αυτό, αποφάσισε να συντάξει το παρών κεφάλαιο με λίγες παραπάνω κουβέντες.

\section{Παραδοχές και ανάλυση δεδομένων}

\noindent Κύριος στόχος των πειραματικών μετρήσεων ήταν ο προσδιορισμός του τοπικού αδιάστατου αριθμού Nusselt, $\overline{Nu}$, κατά μήκος του όγκου ελέγχου του συστήματος, δηλαδή στην περιοχή που περιβάλλεται από τον εξωτερικό και εσωτερικό κύλινδρο. Με αυτό ως γνώμονα, αναπτύχθηκε αναλυτική μεθοδολογία προσδιορισμού των σχετικών μεγεθών, βασισμένη σε ορισμένες παραδοχές που αφορούν (i) το πεδίο ροής και (ii) την πειραματική διάταξη.

Οι προϋποθέσεις που πρέπει να ικανοποιούνται στο πεδίο ροής είναι οι εξής:

\begin{itemize}
\item Το εργαζόμενο μέσο είναι αέρας και εισέρχεται στην διάταξη με θερμοκρασία ίση με αυτή του περιβάλλοντος.
\item Το εργαζόμενο μέσο είναι Νευτωνικό ρευστό και συμπεριφέρεται ως ιδανικό αέριο.
\item H θερμοκρασία του αέρα, μεταξύ εισόδου και εξόδου της διάταξης, συμβαίνει γραμμικά. Η παραδοχή αυτή είναι συνήθης σε ροές κυκλικών διατομών για οριακές συνθήκες σταθερής ροής θερμότητας της θερμαινόμενης επιφάνεια \cite{2011_Bergman_BOOK}.
\item Η ειδική θερμότητα $\left(C_p\right)$ λήφθηκε \qty{1.005}{\kilo\joule\per\kilogram\per\kelvin}, η μέση τιμή μεταξύ θερμοκρασιών 250 και \qty{350}{\kelvin} \cite{1955_Hilsenrath_BOOK} - στο πείραμα η θερμοκρασία του αέρα κυμάνθηκε από 298 μέχρι \qty{329}{\kelvin}. Οι υπόλοιπες θερμοδυναμικές ιδιότητες του ρευστού $\left(\rho, \, \kappa, \mu \right)$ ελήφθησαν για τη μέση τιμή θερμοκρασιών εισόδου και εξόδου, για ατμοσφαιρική πίεση μίας ατμόσφαιρας. Αυτή είναι μια καλή πρακτική για θερμοκρασιακές διαφορές θερμαινόμενης επιφάνειας-αέρα μικρότερες των \qty{20}{\degreeCelsius} σύμφωνα με τους \citeauthor{1993_Kays_BOOK} \cite{1993_Kays_BOOK}. Αναλυτικότερα, οι θερμοδυναμικές ιδιότητες του αέρα υπολογίστηκαν ως εξής:

\begin{itemize}
\item η πυκνότητα $\rho$, για μέση θερμοκρασία εισόδου-εξόδου $T_{avg.}$, υπολογίστηκε από της εξίσωση ιδανικών αερίων \cite{Φιλιός2020}:$$\rho = \displaystyle\frac{P}{R T_{avg}}$$
\item η θερμική αγωγιμότητα αέρα $\kappa_{\text{αέρα}}$, για μέση θερμοκρασία εισόδου-εξόδου $T_{avg.}$, υπολογίστηκε από την εμπειρική εξίσωση των \citeauthor{1951_Kannuluik} \cite{1951_Kannuluik}:$$\kappa \simeq \num{5.75e-5}\left(1 + \num{317e-5}  T_{avg.} - \num{21e-7} T_{avg.} ^ 2\right) 418.4$$
\item το δυναμικό ιξώδες $\mu$, για μέση θερμοκρασία εισόδου-εξόδου $T_{avg.}$, υπολογίστηκε από την εμπειρική εξίσωση Sutherland \cite{Φιλιός2020}:$$\mu \simeq \num{1.716e-6} \left(\displaystyle\frac{T_{avg.} + 273.15}{273.15}\right) ^ 2 + \left(\displaystyle\frac{273.15 + 110.56}{T_{avg.} + 110.56}\right)$$
\end{itemize}

\item Η ροή θεωρείται ασυμπίεστη. Εδώ υιοθετείται η ακριβής έννοια\footnote{είθισται, με τη διασταλτική έννοια του όρου, λέγοντας ασυμπίεστη ροή να εννοείται ότι η πυκνότητα είναι ανεξάρτητη όλων των παραγόντων και όχι μόνο της πίεσης \cite{Ανδρέας2021}} του όρου, δηλαδή ότι η πυκνότητα της ροής είναι ανεξάρτητη από την πίεση αλλά δύναται να επέλθουν διαφοροποιήσεις λόγο μοριακής αγωγιμότητας της θερμότητας του ρευστού~\footnote{για να είναι το πεδίο ροής ασυμπίεστο, δηλαδή να είναι το πεδίο ταχύτητας σωληνοειδές, θα πρέπει να πληρούνται πέντε κριτήρια σύμφωνα με τον \citeauthor{2000_Batchelor_BOOK} \cite{2000_Batchelor_BOOK}, σελ. 75 και 167 - 171}. Ουσιαστικά, και αυτό είναι ερμηνεία του συγγραφέα, η πυκνότητα μεταβάλλεται αλλά προσεγγιστικά (προσέγγιση Boussinesq) ισχύει $\nabla \cdot U = 0$.
\item Η ροή είναι μόνιμη, οπότε όλα τα χαρακτηριστικά μεγέθη του πεδίου ροής είναι ανεξάρτητα του χρόνου.
\item Το πάχος του υδροδυναμικό οριακό στρώμα $\left(\delta\right)$, σε όλο το πεδίο ροής, συμπίπτει με αυτό του θερμικού οριακού στρώματος, δηλαδή $Pr = 1$. Αυτό σημαίνει πολύ απλά ότι η μεταφορά ενέργειας και ορμής, μέσω του μηχανισμού της διάχυσης, είναι συγκρίσιμες.
\item H ροή, στο σημείο που εξασθενεί η περιδίνησή της, είναι στρωτή. Αυτό βρίσκει σύμφωνη τη βιβλιογραφία \cite{Dou2005} για εύρος αριθμού Reynolds 1100 - 2000, και για διάταξη ομόκεντρων κυλίνδρων των οποίων το πάχος δακτυλίου $\left(R_{\text{εξ.}}\right.$ - $\left. R_{\text{εσ.}}\right)$ ισούται με τη διάμετρο οπής βρόγχων $\left(D_\beta\right)$. Αυτού του είδους η ροή είναι γνωστή και ως "pure swirl flow" \cite{1991_Legentilhomme}.
\end{itemize}

Ενώ οι προϋποθέσεις που πρέπει να πληροί η πειραματική εγκατάσταση είναι:

\begin{itemize}
\item Η θέρμανση του εσωτερικού κυλίνδρου γίνεται αξονοσυμμετρικά, αυτό σημαίνει πολύ απλά ότι η αντίσταση θερμαίνεται ομοιόμορφα ως προς τη γωνιακή συνιστώσα $\theta$ ή διαφορετικά $\partial T_{\text{αντ.,x}} / \partial \theta = 0$.
\item Οι τοπικές θερμοκρασίες του εσωτερικού κυλίνδρου είναι ενδεικτικές της επιφάνειάς του αλλά και της μέσης εσωτερικής θερμοκρασίας του. Θεωρείται ουσιαστικά ότι η κλίση της εσωτερικής θερμοκρασίας $\nabla T_x$ είναι αρκετά μικρότερα της θερμοκρασιακής διαφοράς $T_{\text{αντ.,x}} - T_{\text{αέρα,x}}$ που χρησιμοποιείται στον προσδιορισμό του τοπικού συντελεστή συναγωγής $h_x$. Αυτό είθισται να ποσοτικοποιείται με τον αριθμό Biot, $hr/k$,  ο οποίος συνήθως ορίζεται να είναι μικρότερος του 0.01 (ανάλογα με το πόσο κρίσιμο είναι το πείραμα φυσικά) \cite{1985_Moffat}.
\item Η διαδικασία θεωρείται αδιαβατική, δεν υπάρχουν απώλειες λόγω ακτινοβολίας και αγωγής/συναγωγής του εξωτερικού κυλίνδρου. Το σύστημα θεωρείται τέλεια μονωμένο.
\end{itemize}

Για κάθε σενάριο που μελετήθηκε, το πείραμα έτρεχε για περίπου 90 λεπτά προτού γίνουν οποιεσδήποτε μετρήσεις, έτσι ώστε να εξασφαλιστεί σταθερή ροή θερμότητας στην αντίσταση, $\dot{q}_{conv.}$. Αυτό επιτεύχθηκε με τον συνεχή έλεγχο των ενδείξεων των θερμοστοιχείων μέχρι οι τιμές τους να συγκλίνουν προς μία σταθερή τιμή.

Τα χαρακτηριστικά μεγέθη της πειραματικής διάταξης αναγράφονται στον Πίνακα~\ref{apparvalues}.

\begin{table}[!htbp]
\caption{Χαρακτηριστά μεγέθη πειραματικής διάταξης}
\centering
\label{apparvalues}
\ra{1.3}
\begin{threeparttable}
\begin{tabular}{@{}lccc@{}}
\toprule
Μέγεθος & Σύμβολο & Διαστάσεις & Τυπική τιμή \\
\midrule
Διάμετρος εξωτερικού κυλίνδρου & $D_{\text{εξ.}}$ & \unit{\metre} & \num{40e-3}\\

Διάμετρος εσωτερικού κυλίνδρου & $D_{\text{εσ.}}$ & \unit{\metre} & \num{22e-3}\\

Διάμετρος οπής βρόγχων & $D_{\beta}$ & \unit{\metre} & \num{9e-3}\\

Μήκος εσωτερικού κυλίνδρου & $L$ & \unit{\metre} & 0.9\\

Γωνία βρόγχων & $\phi$ & \unit{\degree}\tnote{*} & \ang{45}, \ang{60}, \ang{75} και \ang{90}\\

Αριθμός βρόγχων & $\alpha$ & \dots & 1, 2, 3 και 4\\
\bottomrule
\end{tabular}
\smallskip
\begin{tablenotes}
\item[*] \footnotesize{Δεν είναι μονάδα SI, όμως το επίσημο εγχειρίδιο \enquote{\textit{Le Système international d’unités (SI)}} \cite{Brochure2019} λέει εμφατικά \enquote{accepted non-SI unit,} σελ. 145–146}
\end{tablenotes}
\end{threeparttable}
\end{table}

Για την αξιολόγηση της αποτελεσματικότητας μεταφοράς θερμότητας χρησιμοποιήθηκαν οι μετρήσεις των εννέα, ομοιόμορφα τοποθετημένων, θερμοστοιχείων κατά μήκος της αντίστασης. Ως εκ τούτου, το σύστημα χωρίστηκε σε εννέα επιμέρους υποσυστήματα, καθένα από τα οποία θα έχει το δικό του αντιπροσωπευτικό θερμοστοιχείο. Κατ' επέκταση ο όγκος ελέγχου θα χωρίστηκε σε εννέα επιμέρους όγκους. Στο \prettyref{fig:controlvolume} φαίνεται η σχετική αναπαράσταση όσων ειπώθηκαν.

Ο αέρας εισέρχεται από το υποσύστημα 1, το κάτω μέρος της πειραματικής διάταξης, και απομακρύνεται από το υποσύστημα 9, το πάνω μέρος της πειραματικής διάταξης. Η $T_{\infty}$ είναι η θερμοκρασία του αέρα που εισέρχεται στο σύστημα και $T_{\text{εξ.}}$ η θερμοκρασία του αέρα που εξέρχεται αυτού. H αντιπροσωπευτική θερμοκρασία της αντίστασης κάθε υποσυστήματος  συμβολίζεται ως $T_{\text{αντ.,x}}$, ενώ η θερμοκρασία του αέρα, για το ύψος $z$ που είναι τοποθετημένο το έκαστοτε θερμοστοιχείο, συμβολίζεται ως $T_{\text{αέρας,x}}$.

\begin{figure}[hbp]
\centering
\subimport{Chapter4/Figs/pdftex}{controlvolume.pdf_tex}
\caption{Όγκος ελέγχου πειραματικών διατάξεων}
\label{fig:controlvolume}
\end{figure}

\clearpage

\subsection{Εκτίμηση θερμοκρασίας αέρα}

\noindent Όπως και αναφέρθηκε προηγουμένως, για οριακές συνθήκες σταθερής ροής θερμότητας εκ μέρους της αντίστασης, η θερμοκρασία του αέρα, μεταξύ εισόδου και εξόδου της διάταξης, εξελίσσεται γραμμικά \cite{2011_Bergman_BOOK}. Η θερμοκρασία του αέρα, οπότε, υπολογίζεται από:

\begin{equation}\label{eq:tempair}
T_{\text{αέρα,x}} = \left(\displaystyle\frac{T_{\text{εξ.}} - T_{\infty}}{0.9}\right)z + T_{\infty}
\end{equation}


\subsection{Υπολογισμός ογκομετρικής παροχής}

\noindent Η ογκομετρική παροχή υπολογίστηκε λαμβάνοντας τον χρόνο που χρειάστηκαν \qty{0.1}{\metre\cubed} αέρα να εξέλθουν από την πειραματική διάταξης. Η εξίσωση που χρησιμοποιήθηκε άρα ήταν η εξής: 

\begin{equation}\label{eq:floww}
\dot{\volume} = \frac{\volume}{t}
\end{equation}

\subsection{Υπολογισμός ηλεκτρικής ισχύος αντίστασης και ανεμιστήρα}

\noindent Η ηλεκτρική ισχύς στο παρών πείραμα ήταν δίσημη, και αυτό διότι χρειάστηκε για θέρμανση της αντίστασης (δηλαδή του εσωτερικού κυλίνδρου) και την τροφοδότηση του ανεμιστήρα αναρρόφησης. Η τιμής της πρώτης παρέμεινε σταθερή καθ' όλη τη διάρκεια των πειραμάτων - για να καταστεί δυνατή η σύγκριση μεταξύ των διαφόρων διατάξεων - με τιμές ρεύματος \qty{0,63}{\ampere} και τάσης \qty{29.5}{\volt}, οι οποίες δίνουν την παρεχόμενη ηλεκτρική ισχύ της αντίστασης:

\begin{equation*}
\dot{Q}_{res.} = \qty{18.85}{\watt}
\end{equation*}

\noindent Εν αντιθέσει, η ηλεκτρική ισχύς για την τροφοδότηση του ανεμιστήρα αναρρόφησης, μεταβάλλεται για διάφορες τιμές της παροχής. Η καταναλισκόμενη ισχύς βρίσκεται από:

\begin{equation}\label{eq:fanpower}
\dot{W} = V_{\text{ανεμ.}} A_{\text{ανεμ.}}
\end{equation}

\subsubsection{Αντιπροσωπευτική τιμή ισχύος}\label{reppower}

\noindent Έχοντας τις τιμές ηλεκτρικής ισχύος $\dot{W}$, για τέσσερις τιμές παροχής, μπορούμε πλέον να εφαρμόσουμε παλινδρόμηση δύναμης ούτως ώστε να εξαχθεί σχέση συσχέτισης για συνθήκες ροής 1100 με 2000 Reynolds. Τα υπό εξέταση δεδομένα φέρουν αβεβαιότητες, με πιο επικρατούσα να είναι αυτή της παροχής (βλ. υποενότητες \ref{powerunc} και \ref{flowunc}), ως εκ τούτου χρησιμοποιήθηκε ο \prettyref{code:lsqfit} ο οποίος βασίζεται στη μέθοδο των ελαχίστων τετραγώνων, και η μεθοδολογία του οποίου αναπτύσσεται διεξοδικά στο \prettyref{app:lsqAppendix}.

Οπότε για τα τέσσερα ζεύγη δεδομένων $\left(\dot{W} \pm \sigma_{\scriptsize{\dot{w}}}, \, \dot{\volume} \pm \sigma_{\scriptsize{\dot{\volume}}}\right)$ βρίσκεται:

\begin{equation}\label{functionwq}
\dot{W} = a\dot{\volume}^b
\end{equation}

\noindent όπου $a$ και $b$, σταθερές της συσχέτισης δύναμης.

Η αντιπροσωπευτική τιμή ισχύος μπορεί να υπολογισθεί, πλέον, από το θεώρημα μέσης τιμής της ανάλυσης συναρτήσεων:

\begin{equation}\label{pwrep}
\dot{W}_{\scriptsize{avg}} = \displaystyle\frac{1}{\num{1.6d-3} - \num{0.8d-3}} \int_{\num{0.8d-3}}^{\num{1.6d-3}} a{\dot{\volume}}^b d\left(\dot{\volume}\right)
\end{equation}

\noindent Η τιμή $\dot{W}_{\scriptsize{avg}}$ υπολογίστηκε αριθμητικά με τη συνάρτηση ολοκλήρωσης \footnote{\href{https://se.mathworks.com/help/matlab/ref/trapz.html}{trapz}} \cite{matlabtrapz} του προγραμματιστικού περιβάλλοντος \matlab. Όσον αφορά τη διάδοση του σφάλματος στον όρο $\dot{W}_{\scriptsize{avg}}$, η μεθοδολογία που αναπτύσσεται στην συνέχεια (βλ. υποενότητα \ref{uncresult}) δεν μας καλύπτει. Ως εκ τούτου, χρησιμοποιήθηκε η στατιστική εργαλειοθήκη\footnote{\href{https://www.mathworks.com/products/curvefitting.html}{Curve Fitting toolbox}} \cite{matlabcurvefitting} του \matlab, όπου εκτιμούνται τα άνω και κάτω όρια της συνάρτησης \ref{functionwq}, και εν συνεχεία ολοκληρώνονται για να χρησιμοποιηθούν ως σχετικές αβεβαιότητες. Ακολουθεί απόσπασμα από τον \prettyref{code:datapadding} που πραγματεύεται ακριβώς αυτό:

\begin{lstlisting}[firstnumber=561]
% Υπολογισμός αντιπροσωπευτικής ισχύος, και του αντίστοιχου σφάλματος, για κάθε
% διάταξη
g1 = fit(flowdata', wattdata', 'power1');
a = g1.a;
b = g1.b;
    
func = @(x) a*x^b;

% Συνεισφορά αβεβαιότητας ογκομετρικής παροχής    
uP = zeros(1, 4);
for i = 1:4
    [Puu, uP(1, i)] = UncertaintyPropagation(func, flowdata(1, i), flowerr(1, i)); 
end
 
% Συνολικά βάρη συνάρτησης   
error = sqrt(watterr .^ 2 + uP .^ 2);
weights = 1 ./ error .^ 2;

g2 = fit(flowdata', wattdata', 'power1', 'weight', weights);

yhat = g2.a * exp(qFit * g2.b);
CIO = predint(g2, qFit, 0.95, 'obs');

% Υπολογισμός αντιπροσωπευτικής Ισχύος και των αβεβαιοτήτων της
relWattFan(1, 1, k) = trapz(qFit, yhat) / (max(qFit) - min(qFit));
relWattFanErr(1, 1, k) = trapz(qFit, CIO(:, 2)) / (max(qFit) - min(qFit)) - trapz(qFit, yhat) / (max(qFit) - min(qFit));
\end{lstlisting}

\subsection{Εκτίμηση θερμοκρασιακής ομοιογένειας}

\noindent Στο πλαίσιο της θερμοδυναμικής σύγκρισης μεταξύ των διαφόρων διατάξεων, κρίθηκε σκόπιμη η εκτίμηση της θερμοκρασιακής ομοιογένειας - δηλαδή το πόσο ομοιόμορφο είναι το θερμοκρασιακό προφίλ σε κάθε πείραμα. Για να γίνει αυτό, χρησιμοποιήθηκαν οι στατιστική όροι της μέσης τιμής και της τυπικής απόκλισης.

Η μέση τιμή θερμοκρασίας $Τ_{mean}$ θα δώσει μία αρχική εκτίμηση της αντιπροσωπευτικής θερμοκρασίας στην οποία ισορροπεί κάθε πειραματική διάταξη, για διάφορες τιμές παροχής, ενώ η τυπική απόκλιση $T_{\sigma}$  θα επιδείξει την εγγύτητα των διαφορετικών θερμοκρασιών ως προς τη μέση αυτή τιμή. Ποσοτικοποιείται κατ' αυτόν τον τρόπο το εύρος θερμοκρασιών κάθε διάταξης, και μπορούμε να εξάγουμε συμπεράσματα ως προς την ομοιομορφία ψύξης. Μία μικρή τιμή $T_{\sigma}$, για παράδειγμα, σημαίνει ότι το εύρος θερμοκρασιών είναι σχετικά μικρό, και άρα, η εν λόγω διάταξη υπόκειται σε μικρότερες θερμικές καταπονήσεις.

Τα θερμοστοιχεία διέπονται από διαφορετικές αβεβαιότητες όμως, άρα κάθε μέτρηση θα πρέπει να έχει διαφορετική βαρύτητα αναλόγως του σφάλματός της. Οπότε, για την συνεισφορά του κάθε θερμοστοιχείου σε μια συνολική τιμή μέσης θερμοκρασίας $T_{mean}$, χρησιμοποιείται ο σταθμισμένος μέσος \parencites{1997_Taylor_BOOK}{2006_James_BOOK}:

\begin{equation}\label{eq:wmeantemp}
T_{mean} = \displaystyle\frac{\displaystyle\sum_{x=1}^{9} w_x T_{\text{αντ,x}}}{\displaystyle\sum_{x=1}^{9} w_x}
\end{equation}

\noindent όπου

\begin{equation*}
w_x = \displaystyle\frac{1}{\left(\delta T_x\right)^2}
\end{equation*}

\noindent Η διασπορά του σταθμισμένου μέσου δίνεται από \parencites{2003_Bevington_BOOK}{Kirchner2006}:

\begin{equation}\label{eq:varwmean}
\left(s_T ^2 \right)_{wtd} = \displaystyle\frac{\displaystyle\sum_{x=1}^{9} w_x \left(T_{\text{αντ,x}} - T_{mean}\right)^2}{\displaystyle\sum_{x = 1}^{9} w_x} \, \displaystyle\frac{n}{n-1}
\end{equation}

\noindent και αξιοποιώντας το θεώρημα μέσης τιμής, η σταθμισμένη τυπική απόκλιση θα είναι:

\begin{equation}\label{eq:errorwmean}
T_{\sigma} = \displaystyle\sqrt{\frac{\left(s_T ^2 \right)_{wtd}}{n}}
\end{equation}

\noindent Οι σταθμισμένοι μέσος και τυπική απόκλιση υπολογίστηκαν αξιοποιώντας τον \prettyref{code:wvar}.

\subsection{Εκτίμηση αριθμού Reynolds}

\noindent Έχοντας υπολογίσει την παροχή, μπορεί να προσδιοριστεί ο αριθμός Reynolds κάθε ροής. Ο αδιάστατος αριθμός αυτός ισούται με το λόγο αδρανειακών και ιξωδών δυνάμεων, και χρησιμοποιείται για τον προσδιορισμό του είδους ροής. Λαμβάνοντας υπόψη τον τύπο της υδραυλικής διαμέτρου για το εν λόγω σύστημα:

\begin{equation*}
D_h = \displaystyle\frac{\displaystyle\frac{4 \pi \left(D_{\text{εξ.}}^2 - D_{\text{εσ.}}^2\right)}{4}}{\pi \left(D_{\text{εξ.}} + D_{\text{εσ.}}\right)} = D_{\text{εξ.}} - D_{\text{εσ.}}
\end{equation*}

\noindent και τη μέση ταχύτητα ροής,

\begin{equation*}
\overline{U} = \displaystyle\frac{\dot{\volume}}{A_{ann.}}, \, \text{όπου} \quad A_{ann.} = \displaystyle\frac{\pi \left(D_{\text{εξ.}}^2 - D_{\text{εσ.}}^2\right)}{4}
\end{equation*}

\noindent ο αριθμός Reynolds θα είναι:

\begin{equation}\label{eq:reynolds}
Re = \displaystyle\frac{\rho \, \overline{U} D_h}{\mu} = \displaystyle\frac{4 \, \rho \, \dot{\volume}}{\pi \mu \left(D_{\text{εξ.}} + D_{\text{εσ.}}\right)}
\end{equation}

\subsection{Εκτίμηση αριθμού Nusselt}

\noindent Η πιο σύνηθες έκφραση για τον προσδιορισμό της αποτελεσματικότητας της ενίσχυσης μεταφοράς θερμότητας, μέσω εξαναγκασμένης ροής, έχει καθιερωθεί να είναι o αδιάστατος αριθμός Nusselt (Nu). Ο αριθμός αυτός εκφράζει την ενίσχυση μεταφοράς θερμότητας λόγω συναγωγής σε σύγκριση με την αγωγή, και είναι συνήθως συνάρτηση των γεωμετρικών χαρακτηριστικών του συστήματος $\left(x, L\right)$ και των αριθμών Reynolds και Prandtl:

\begin{equation*}
{Nu}_x = f\left(x, {Re}_x, Pr\right)
\end{equation*}

\noindent η οποία για την παραδοχή $Pr = 1$ γίνεται:

\begin{equation}\label{corellation}
{Nu}_x = f\left(x, {Re}_x\right)
\end{equation}

\noindent Στην παρούσα εργασία θέλουμε να εξάγουμε έναν αντιπροσωπευτικό Nusselt για κάθε διάταξη και για εύρη συνθηκών ροής Reynolds 1100 με 2000. Για να επιτευχθεί αυτό, θα λειτουργήσουμε αποσπασματικά: (i) θα υπολογιστούν οι τοπική αριθμοί Nusselt που αντιστοιχούν στις θέσεις που έχουν τοποθετηθεί τα θερμοστοιχεία κάθε διάταξης, (ii) θα υπολογιστεί ο μέσος αριθμός Nusselt από στατιστική ανάλυση των επιμέρους τοπικών αριθμών Nusselt, και (iii) θα γίνει παλινδρόμηση δύναμης για να οριστεί συνάρτηση συσχέτισης, όπως αυτή της \ref{corellation}, από την οποία θα εκτιμηθεί ο αντιπροσωπευτικός αριθμός Nusselt με χρήση αριθμητικής μεθόδου ολοκλήρωσης.

\subsubsection{Τοπικός Nusselt}
\noindent Ο τοπικός αριθμός Nusselt εκφράζεται ως:

\begin{equation}\label{eq:localnusselt}
Nu_x = \displaystyle\frac{h_x D_h}{k_{\text{αέρα}}} = \displaystyle\frac{h_x \left(D_{\text{εξ.}} - D_{\text{εσ.}}\right)}{k_{\text{αέρα}}} 
\end{equation}

\noindent όπου $h_x$ ο τοπικός συντελεστής συναγωγής, $D_h$ η υδραυλική διάμετρος της διάταξης και $k_{\text{αέρα}}$ η θερμική αγωγιμότητα του αέρα. Εφαρμόζοντας ενεργειακό ισοζύγιο σε απειροστά μικρή απόσταση από την επιφάνεια του εσωτερικού κυλίνδρου, ο τοπικός συντελεστής συναγωγής, με τη σειρά του, δίνεται από:

\begin{align}\label{eq:localhtc}
h_x = \displaystyle\frac{\dot{q}_{conv.}}{T_{\text{αντ., x}} - T_{\text{αέρα, x}}} &= \displaystyle\frac{\dot{Q}_{res.}}{A_{heated} \left(T_{\text{αντ., x}} - T_{\text{αέρα, x}}\right)} \nonumber\\[2pt] &= \displaystyle\frac{V_{\text{αντ.}} Α_{\text{αντ.}}}{\displaystyle\frac{\pi D_{\text{εσ.}} L}{0.9} \left(T_{\text{αντ., x}} - T_{\text{αέρα, x}}\right)}
\end{align}

\subsubsection{Μέσος αριθμός Nusselt}
\noindent Όπως στην περίπτωση των θερμοστοιχείων, έτσι κι εδώ, οι $Nu_x$ διέπονται από διαφορετικές αβεβαιότητες, άρα κάθε υπολογισμός θα πρέπει να έχει διαφορετική βαρύτητα αναλόγως το σφάλμα του. Οπότε, για τον συνδυασμό του κάθε $Nu_x$ σε μια συνολική τιμή $\overline{Nu}$, χρησιμοποιείται ο σταθμισμένος μέσος \parencites{1997_Taylor_BOOK}{2006_James_BOOK}:

\begin{equation}\label{eq:nussavg}
\overline{Nu} = \displaystyle\frac{\displaystyle\sum_{x=1}^{9} w_x Nu_x}{\displaystyle\sum_{x=1}^{9} w_x}
\end{equation}

\noindent όπου

\begin{equation*}
w_x = \displaystyle\frac{1}{\left(\delta Nu_{x}\right)^2}
\end{equation*}

\noindent Η διασπορά του σταθμισμένου μέσου δίνεται από \cite{2003_Bevington_BOOK}:

\begin{equation}\label{eq:nussvar}
\left(s_{Nu} ^2 \right)_{wtd} = \displaystyle\frac{\displaystyle\sum_{x=1}^{9} w_x \left(Nu_x - \overline{Nu}\right)^2}{\displaystyle\sum_{x = 1}^{9} w_x} \, \displaystyle\frac{n}{n-1}
\end{equation}

\noindent και αξιοποιώντας το θεώρημα μέσης τιμής, η σταθμισμένη τυπική απόκλιση θα είναι:

\begin{equation}\label{errornu}
\sigma_{\overline{Nu}} = \displaystyle\frac{\left(s_{Nu} ^2 \right)_{wtd}}{\sqrt{n}}
\end{equation}

\noindent Η τυπική απόκλιση της \ref{errornu} θα χρησιμοποιηθεί στους υπολογισμούς του αντιπροσωπευτικού Nusselt.

\subsubsection{Αντιπροσωπευτικός αριθμός Nusselt}\label{repnu}

\noindent Όπως και στον υπολογισμό της αντιπροσωπευτικής τιμής ισχύος ($\dot{W}_{\scriptsize{avg}}$), έτσι κι εδώ, έχουμε τους μέσους αριθμούς Nusselt $\overline{Nu}$, για τέσσερις τιμές Reynolds, και άρα μπορούμε πλέον να εφαρμόσουμε παλινδρόμηση δύναμης ούτως ώστε να εξαχθεί σχέση συσχέτισης για συνθήκες ροής 1100 με 2000 Reynolds.

Οπότε για τα τέσσερα ζεύγη δεδομένων $\left(Re \pm \sigma_{\scriptsize{Re}}, \, \overline{Nu} \pm \sigma_{\scriptsize{\overline{Nu}}}\right)$ και βρίσκεται:

\begin{equation*}
\overline{Nu} = a{Re}^b
\end{equation*}

\noindent όπου $a$ και $b$, σταθερές της συσχέτισης δύναμης. Να σημειωθεί ότι ίδια συσχέτιση έχει χρησιμοποιηθεί υπό ίδιες παραδοχές, και σε παρόμοια γεωμετρία, και σε άλλες έρευνες \parencites{1975_Hay}{1971_Narezhnyy}{1967_Blum_CONF}{2017_Rao}.

Ο αντιπροσωπευτικός αριθμός Nusselt μπορεί να υπολογισθεί, πλέον, από το θεώρημα μέσης τιμής ανάλυσης συναρτήσεων:

\begin{equation}\label{nusepp}
\overline{Nu}_{\scriptsize{avg}} = \displaystyle\frac{1}{2000 - 1100} \int_{1100}^{2000} a{Re}^b d\left(Re\right)
\end{equation}

\noindent Η τιμή $\overline{Nu}_{\scriptsize{avg}}$ καθώς επίσης και στο σφάλμα της, $\sigma_{\overline{\mbox{\scriptsize{Nu}}}\mbox{\tiny{avg}}}$, υπολογίστηκαν με τον ίδιο ακριβώς τρόπο που υπολογίστηκαν  τα $\dot{W}_{\scriptsize{avg}}$, $\sigma_{\scriptsize{\dot{W}}\scriptsize{avg}}$ (βλ. υποενότητα \ref{reppower}).

\subsection{Υπολογισμός ροής θερμότητας προς τον αέρα}

\noindent Η ροή θερμότητας προς τον αέρα κάθε διάταξης, δηλαδή ο ρυθμός ψύξης της αντίστασης από τον αέρα, υπολογίζεται από την εξίσωση:

\begin{equation}\label{eq:htrate}
\dot{Q}_{sys.} = \dot{m} \, C_p \, \Delta T = \rho \, \dot{\volume} \, C_p \left(Τ_{\text{εξ.}} - T_{\infty}\right)
\end{equation}
 
\noindent όπου $\dot{m}$ η ροή μάζας αέρα, $\rho$ η πυκνότητα του αέρα, $C_p$ η ειδική θερμοχωρητικότητα του αέρα και $T_{\infty}, \, Τ_{\text{εξ.}}$ οι θερμοκρασίες εισόδου και εξόδου του αέρα αντίστοιχα.
 
\subsection{Υπολογισμός δεικτών βελτίωσης}

\noindent Τα αποτελέσματα κάθε πειραματικής διάταξης θα παρουσιάζουν τις δικές τους ιδιομορφίες, προφανώς λόγω των διαφορετικών παραμέτρων $\left(\alpha, \phi \right)$ που χρησιμοποιήθηκαν για κάθε περίπτωση. Για να διευκολυνθεί λοιπόν η σύγκριση, στις περιπτώσεις περιδινούμενης ροής με την αντίστοιχη της αξονικής ροής, θα χρησιμοποιηθούν δείκτες οι οποίοι θα υποδηλώνουν τη (i) θερμική, (ii) την ενεργειακή και (iii) τη δυναμική βελτίωση. 

\subsubsection{Δείκτης βελτίωσης θερμικής μεταφοράς}

\noindent Η βελτίωση μετάδοσης θερμότητας, για κάθε διάταξη περιδινούμενης ροής, σε σύγκριση με την αντίστοιχη αξονική, δίνεται από:

\begin{equation}\label{eq:htii}
HTII = \displaystyle\frac{\overline{Nu}_{\scriptsize{avg, \alpha, \phi}} - \overline{Nu}_{\scriptsize{avg, axial}}}{\overline{Nu}_{\scriptsize{avg, axial}}} \times 100
\end{equation}

\noindent όπου $\alpha$ = αριθμός βρόγχων, $\phi$ = γωνία βρόγχων, $axial$ = αξονική ροή. Η σημασία είναι η ίδια στους δύο ακόλουθους δείκτες βελτίωσης.

\subsubsection{Δείκτης βελτίωσης καταναλισκόμενης ισχύος}

\noindent Η βελτίωση καταναλισκόμενης ισχύος, για την τροφοδοσία ανεμιστήρα, για κάθε διάταξη περιδινούμενης ροής, σε σύγκριση με την αντίστοιχη αξονική, δίνεται από:

\begin{equation}\label{eq:pii}
PII = \displaystyle\frac{\dot{W}_{\scriptsize{avg, axial}} - \dot{W}_{\scriptsize{avg, \alpha, \phi}}}{\dot{W}_{\scriptsize{avg, axial}}} \times 100
\end{equation}

\noindent Να σημειωθεί ότι οι κάτω δείκτες είναι ανεστραμμένοι με τους αντίστοιχους των $HTII$,  $PEI$. Αυτό συμβαίνει διότι, στην κατανάλωση ισχύος, ως βελτίωση θεωρείται η μείωση της απαιτούμενης ισχύος.

\subsubsection{Δείκτης βελτίωσης ωφέλιμου δυναμικού} 
 
\noindent Το ωφέλιμο δυναμικό, στην προκειμένη περίπτωση, ορίζεται ως το πόση ροή θερμότητας έλαβε ο αέρα για δεδομένη παροχή ισχύος του ανεμιστήρα. Η βελτίωση του ωφέλιμου δυναμικού δίνεται, λοιπόν, για κάθε διάταξη περιδούμενης ροής, σε σύγκριση με την αντίστοιχη αξονική, από:  
 
\begin{equation}\label{eq:pei}
PEI = \displaystyle\frac{\left( \frac{\dot{Q}_{sys.}}{\dot{W}_{\scriptsize{avg}}} \right)_{\alpha, \phi} - \left( \frac{\dot{Q}_{sys.}}{\dot{W}_{\scriptsize{avg}}} \right)_{axial}}{\left( \frac{\dot{Q}_{sys.}}{\dot{W}_{\scriptsize{avg}}} \right)_{axial}} \times 100
\end{equation}  
 
\section{Ανάλυση αβεβαιότητας ατομικών μετρήσεων}\label{uncertaintyofmeasurements}

\noindent Στα πραγματοποιηθέντα πειράματα, το χρονικό διάστημα εκτέλεσής τους ήταν χονδρικά μιάμιση ώρα, ούτως ώστε να εξασφαλιστούν οριακές συνθήκες μόνιμης ροής. Αυτό καθιστά την αναπαραγωγή των πειραμάτων, για την εύρεση της επαναναληψιμότητας των μετρήσεων, δυσοίωνη από πλευράς χρόνου. Ως εκ τούτου, χρησιμοποιήθηκε ανάλυση αβεβαιότητας ατομικών μετρήσεων (γνωστή και ως Single-Sample Uncertainty Analysis) \cite{1988_Moffat}.

Η φιλοσοφία της βασίζεται στη συλλογή δεδομένων, για αντιπροσωπευτικό χρόνο λειτουργίας, \textit{a priori} της έναρξης πειραματικών διαδικασιών. Τα δεδομένα αυτά αναλύονται και εξάγονται συμπεράσματα για την συμπεριφορά του συστήματος. Γενικά πρόκειται για χρονοβόρα μεν αλλά αναγκαία (για λόγους πειραματικής συνέπειας) διαδικασία, που περιλαμβάνει τον προσδιορισμό (i) σφάλματος του συστήματος (σφάλμα μηδενικής τάξης), (ii) σφαλμάτων που οφείλονται σε διάφορες μεταβλητές που διέπουν τη λειτουργία του συστήματος (σφάλματα πρώτης και ανωτέρας τάξης) και (iii) σφάλματος που να συμπεριλαμβάνει τα προαναφερθέντα καθώς επίσης και το σφάλμα βαθμονόμησης του οργάνου (γνωστό και ως σφάλμα $N^{\text{ής}}$ τάξης).

Σύμφωνα με τον \citeauthor{1988_Moffat} \cite{1988_Moffat}, η όλη διαδικασία πρέπει να περιλαμβάνει κατ' ελάχιστο τις επιδράσεις που βρίσκονται στα συστηματικά και τα πρώτης τάξεως σφάλματα. Δεδομένου των διαθέσιμων πόρων, και χρόνου, η διαδικασία που ακολουθήθηκε είχε ως εξής:

\begin{enumerate}
\item Προσδιορίστηκε, ή λήφθηκε από σχετικά έντυπα, το σφάλμα ευαισθησίας του μετρητικού, $\delta X_{sens.}$.
\item Προσδιορίστηκε, για κάθε μετρητικό, το σφάλμα μηδενικής τάξης, $\delta X_{0}$, λαμβάνοντας 31 μετρήσεις\footnote{οι 31 μετρήσεις μας εξασφαλίζουν ότι η κατανομή student (βλ. \prettyref{tab:student}) προσεγγίζει την κανονική, και ότι θα λαμβάνουμε τον συντελεστή διόρθωσης, $t_{\scriptsize{30, \qty{95}{\percent}}}$ ίσο με 2.042 για δύο τυπικές αποκλίσεις} ενώ η πειραματική διάταξη βρισκόταν σε ηρεμία, με όσο το δυνατόν σταθερότερες συνθήκες. Για τα δεδομένα αυτά, υπολογίστηκε η τυπική απόκλιση από:
\end{enumerate}

\begin{equation*}
\sigma _X = \left\{\displaystyle\frac{1}{n - 1}\sum_{i=1}^{n} \left(X_{i} - \overline{X}\right) ^ 2\right\} ^ {1/2} \qquad \text{όπου} \qquad \overline{X} = \displaystyle\frac{1}{n}\sum_{i=1}^{31} X_{i} 
\end{equation*}

\noindent και για διάστημα εμπιστοσύνης δύο τυπικών αποκλίσεων, το σφάλμα μηδενικής τάξης υπολογίστηκε:

\begin{equation*}
\delta X_{0} = t_{\scriptsize{n-1,95}} \sigma _X
\end{equation*}

\begin{enumerate}\setcounter{enumi}{2}
\item Τα σφάλματα $\delta X_{sens.}$ και $\delta X_{0}$, καθώς αμφότερα αφορούν το σύστημα, συνδυάζονται μέσω της μεθόδου RSS \cite{2021_Moffat_BOOK}:
\end{enumerate}

\begin{equation*}
\delta X_{0, tot.} = \left\{\left(\delta X_{sens.}\right) ^ 2 + \left(\delta X_{0}\right) ^ 2 \right\} ^ {1/2}
\end{equation*}

\begin{enumerate}\setcounter{enumi}{3}
\item Προσδιορίστηκαν, μόνο για τα θερμοστοιχεία, σφάλματα πρώτης τάξης $\delta X_1$, λαμβάνοντας 31 μετρήσεις σε διάφορες συνθήκες λειτουργίας της πειραματικής διάταξης. Εφαρμόζεται, με μια έννοια, ομαδοποιημένη στατιστική για \enquote{κοντινά} σημεία λειτουργίας του συστήματος υπό την προϋπόθεση ότι η διασπορά σε αυτά θα είναι πραπλήσια \parencites{1973_Abernathy_TECH_REPORT}{2021_Moffat_BOOK}. Η λήψη n αριθμού μετρήσεων, για αριθμό αναπαραγωγών m, δίνει τον ομαδοποιημένο μέσο όρο\parencites{2021_Moffat_BOOK}{2011_Figliola_BOOK}: 
\end{enumerate}

\begin{equation*}
\left\langle\overline{X}\right\rangle = \displaystyle\frac{1}{mn}\sum_{j=1}^{m}\sum_{i=1}^{n} X_{ij}
\end{equation*}

\noindent Η ομαδοποιημένη τυπική απόκλιση βρίσκεται από:

\begin{equation*}
\left\langle\sigma _X\right\rangle = \left\{\displaystyle\frac{1}{m\left(n - 1\right)}\sum_{j=1}^{m}\sum_{i=1}^{n} \left(X_{ij} - \overline{X_j}\right) ^ 2\right\} ^ {1/2} = \left\{\displaystyle\frac{1}{m}\sum_{j=1}^{m} \sigma _{X_j}^2 \right\} ^{1/2}
\end{equation*}

\noindent με βαθμούς ελευθερίας $m(n - 1)$. Η ομαδοποιημένη τυπική απόκλιση των μέσων ορίζεται ως: 

\begin{equation*}
\left\langle\sigma_{\overline{X}}\right\rangle = \displaystyle\frac{\left\langle\sigma _X\right\rangle}{\left( mn\right) ^ {1/2}}
\end{equation*}

\noindent και για διάστημα εμπιστοσύνης δύο τυπικών αποκλίσεων, τελικά έχουμε:

\begin{equation*}
\delta X_1 = t_{\scriptsize{m-1,95}}\left\langle\sigma_{\overline{X}}\right\rangle
\end{equation*}


\begin{enumerate}\setcounter{enumi}{4}
\item Υπολογίστηκε το σφάλμα $N^{\text{ής}}$ τάξης, για κάθε μετρητικό, κάνοντας χρήση του:
\end{enumerate}

\begin{equation}\label{eq:nthunc}
\delta X_N = \left\{\left(\delta X_{0, tot.}\right) ^ 2 +  \left(\delta X_1\right) ^ 2 \right\} ^ {1/2}
\end{equation}

\noindent Η ανωτέρω διαδικασία αυτοματοποιήθηκε με τον \prettyref{code:measurementunc}, εναρμονίζεται στον βασικό κώδικα υπολογισμών (βλ. \prettyref{code:datapadding}), όπου υπολογίζει τα σφάλματα $N^{\text{ής}}$ τάξης των μετρητικών και εν συνεχεία τα αποθηκεύει για περαιτέρω επεξεργασία.

\subsection{Αβεβαιότητα στις μετρήσεις}

\noindent Η ευαισθησία (sensitivity) του χρονομέτρου ήταν \qty{0.01}{\second} ενώ το μέγεθος του σφάλματος μηδενικής τάξης \qty{0.95}{\second} (\prettyref{tab:rep1}), το οποίο για δύο τυπικές αποκλίσεις γίνεται \qty{1.86}{\second}. Η συνολική αβεβαιότητα του χρονομέτρου ήταν λοιπόν:

\begin{equation*}
\delta t = \qty{1.86}{\second}
\end{equation*}

\noindent Οι αβεβαιότητες που σχετίζονται με την ευαισθησία του πολυμέτρου για τιμές τάσης και έντασης ρεύματος ήταν \qty{0.01}{\ampere} και \qty{0.01}{\volt} αντίστοιχα, ενώ τα μεγέθη των σφαλμάτων μηδενικής τάξης \qty{0.11}{\ampere} και \qty{0.01}{\volt} (\prettyref{tab:rep1}). Για δύο τυπικές αποκλίσεις, τα σφάλματα αυτά έγιναν \qty{0.22}{\ampere} και \qty{0.03}{\volt}. Οι συνολικές αβεβαιότητες της τάσης και της έντασης ρεύματος ήταν:

\begin{align*}
\delta V &= \qty{0.22}{\volt}\\
\delta A &= \qty{0.03}{\ampere}
\end{align*}

\noindent Σφάλματα μηδενικής τάξεως (όπως υπολογίστηκαν στην ίδια διάταξη από \citeauthor{2019_Serbes_THESIS} \cite{2019_Serbes_THESIS}) που σχετίζονταν με τη μέτρηση της θερμοκρασίας, για όλα τα θερμοστοιχεία, ήταν σχετικά μικρά, της τάξεως του \qty{0.01}{\degreeCelsius}. Τα αποτελέσματα της ομαδοποιημένης στατιστικής, ωστόσο, ήταν κατά μέσο όρο μία τάξη μεγέθους μεγαλύτερα (\prettyref{tab:rep2}). Τα τελικά σφάλματα των τιμών θερμοκρασίας, για διάστημα εμπιστοσύνης δύο τυπικών αποκλίσεων, εκτιμήθηκαν ως:

\begin{align*}
\delta T_{\text{αντ., 1}} &= \qty{0.14}{\degreeCelsius}, & \delta T_{\text{αντ., 2}} &= \qty{0.17}{\degreeCelsius}, & \delta T_{\text{αντ., 3}} &= \qty{0.18}{\degreeCelsius}\\
\delta T_{\text{αντ., 4}} &= \qty{0.18}{\degreeCelsius}, & \delta T_{\text{αντ., 5}} &= \qty{0.18}{\degreeCelsius}, & \delta T_{\text{αντ., 6}} &= \qty{0.18}{\degreeCelsius}\\
\delta T_{\text{αντ., 7}} &= \qty{0.17}{\degreeCelsius}, & \delta T_{\text{αντ., 8}} &= \qty{0.16}{\degreeCelsius}, & \delta T_{\text{αντ., 9}} &= \qty{0.14}{\degreeCelsius}\\
\delta T_{\text{εξ.}} &= \qty{0.02}{\degreeCelsius}, & \delta T_{\infty} &= \qty{0.01}{\degreeCelsius}
\end{align*}

\noindent Στο υπόλοιπο της εργασίας, όποτε θα γίνεται αναφορά σε σφάλμα, μετρητικού ή αποτελέσματος, θα εννοείται ότι είναι αυτό της $N^{\text{ής}}$ τάξης.

\subsection{Αβεβαιότητα στο αποτέλεσμα}\label{uncresult}

\noindent Δεν είναι λίγες οι περιπτώσεις εκείνες όπου δε γίνεται να μετρηθεί απευθείας η επιθυμητή μεταβλητή κατά την εκτέλεση ενός πειράματος. Οπότε, αντί αυτού, μετριούνται οι τιμές των επιμέρους μεταβλητών της και εν συνεχεία συνδυάζονται με μια εξίσωση συσχέτισης (Data Reduction Equation ή DRE) ώστε να ληφθεί το επιθυμητό αποτέλεσμα.

Η γενική μορφή μιας εξίσωσης συσχέτισης είναι η ακόλουθη:

\begin{equation}\label{eq:DRE}
F = f\left(x_1, x_2, \dots, x_N\right)
\end{equation}

\noindent όπου η μεταβλητή F είναι το αποτέλεσμα συνδυασμού N μεταβλητών.

Για παράδειγμα, στον προσδιορισμό του είδους ροής σε διάταξη ομόκεντρων κυλίνδρων, η εξίσωση που συνδέει τον αριθμό Reynolds με την παροχή είναι:

\begin{equation}\label{eq:Re}
Re = \frac{4 \, \dot{\volume} \, \rho}{\mu \pi \left(D_{\text{εξ.}} + D_{\text{εσ.}}\right)}
\end{equation}

\noindent όπου η μεταβλητή $\rho$ αντιπροσωπεύει την πυκνότητα αέρα, η μεταβλητή $\mu$ το δυναμικό ιξώδες του αέρα, το $\dot{\volume}$ αναπαριστά την παροχή των δοκιμών και $D_{\text{εξ.}},\, D_{\text{εσ.}}$ είναι η οι διάμετροι των εξωτερικού και εσωτερικού κυλίνδρων αντιστοίχως. Κανείς καταλαβαίνει ότι σφάλματα στις μετρήσεις των μεγεθών του δεξιά μέλους στη \prettyref{eq:Re}, θα προκαλέσουν σφάλματα και στην ίδια τη τιμή $Re$ - αυτό ορίζεται ως διάδοση αβεβαιότητας ή propagation of uncertainties.

Σε μία τυπική ανάλυση αβεβαιότητας λοιπόν, ο στόχος είναι να εκφραστεί η συνολική αβεβαιότητα μίας υπολογισμένης μεταβλητής $F$, στο ίδιο διάστημα εμπιστοσύνης με τα αντίστοιχα σφάλματα των επιμέρους συνιστωσών τής $x_i$. Για τον σκοπό αυτό χρησιμοποιήθηκε η μέθοδος γραμμικής διάδοσης του σφάλματος, η οποία βασίζεται στο ανάπτυγμα της σειράς Taylor διατηρώντας τα διαφορικά πρώτης τάξεως\footnote{αυτό γίνεται με τη θεώρηση ότι οι μεταβολές $\delta x_1, \, \delta x_2, \, \dots$ είναι αρκετά μικρές ούτως ώστε τα διαφορικά δεύτερης τάξης ή γινόμενα της μορφής $\left(\delta x_1 \delta x_2\right)$ να είναι ανεπαίσθητα}. H προσέγγιση αυτή, γνωστή και ως Root-Sum-Squared ή RSS, μπορεί να υπολογίσει το συνολικό σφάλμα με αρκετά καλή ακρίβεια, όπως έδειξαν οι \citeauthor{1953_Kline} \cite{1953_Kline}.

Η μέθοδος περιγράφει τη διαδικασία υπολογισμού της ολικής αβεβαιότητας μιας μεταβλητής, υπολογίζοντας τις επιμέρους αβεβαιότητες των συνιστωσών που αποτελούν την μεταβλητή αυτή. Το μέγεθος του υπολογισθέντος σφάλματος είναι επίσης γνωστό και σαν το μέγιστο αναμενόμενο σφάλμα στην υπολογιζόμενη μεταβλητή \cite{1988_Moffat}, και δίνεται από:

\begin{equation}\label{eq:rss}
\delta F = \left\{\left(\frac{\partial F}{\partial x_1} \delta x_1\right) ^ 2 + \left(\frac{\partial F}{\partial x_2} \delta x_2\right) + \dots + \left(\frac{\partial F}{\partial x_N} \delta x_N\right)\right\} ^ {1/2}
\end{equation}

\noindent Η \prettyref{eq:rss} ισχύει εφόσον: 

\begin{enumerate}
\item Οι αβεβαιότητες $\delta x_i$ είναι ανεξάρτητες μεταξύ τους.
\item Η κατανομή των αβεβαιοτήτων $\delta x_i$ είναι παραπλήσια με αυτή κανονικής κατανομής, για κάθε $x_i$.
\item Όλες οι αβεβαιότητες αφορούν το ίδιο διάστημα εμπιστοσύνης - συνήθως $2\sigma \, \text{ή} \, 95\%$.
\end{enumerate}

\begin{flushright}
(όπως αναφέρονται από \citeauthor{1953_Kline} \cite{1953_Kline})
\end{flushright}

Οι προαναφερθείσες υποθέσεις δεν πληρούνται πάντα προφανώς, οπότε ένα εναλλακτικό και πιο στιβαρό σενάριο, για τον υπολογισμό του συνολικού σφάλματος, θα ήταν η προσομοίωση μέσω \textit{Monte Carlo} \cite{2018_HughW.Coleman_BOOK}. Ωστόσο, επειδή απαιτείται αρκετά μεγάλος αριθμός επαναλήψεων για κάθε υπολογισμό (της τάξεως $10^5$), αυτή η προσέγγιση είναι χρονοβόρα και υπολογιστικά απαιτητική.

Ως εκ τούτου, για τους σκοπούς της παρούσας εργασίας, και για λόγους ευκολίας πραγματοποίησής της, θα χρησιμοποιηθεί η μέθοδος RSS για τους υπολογισμούς εκτίμησης συνολικού σφάλματος. 

\subsection{Αυτοματοποιώντας την όλη διαδικασία}\label{uncertpre}

\noindent Η μετάδοση της πειραματικής αβεβαιότητας $\delta x_i$, για τον προσδιορισμό του συνολικού σφάλματος της εκάστοτε συνάρτησης $\delta F$, πραγματοποιείται από τον \prettyref{code:uncertaintyprop}. Ο αλγόριθμος αυτός είναι μια επαναληπτική μέθοδος που επιλύει την \prettyref{eq:rss}, και εν συνεχεία επιστρέφει το συνολικό σφάλμα κάθε υπολογισμού. Παρόμοια ρουτίνα έχει προταθεί από τον \citeauthor{1985_Moffat} \cite{1985_Moffat}.

Μια δυσκολία που προέκυψε σε αυτό το εγχείρημα ήταν οι υπολογισμοί των μερικών παραγώγων της μεταβλητής $F$. Κρίθηκε σκόπιμη λοιπόν, η χρήση μίας υπορουτίνας που να υπολογίζει αριθμητικά τα διαφορικά πρώτης τάξεως.

Η αριθμητική μέθοδος που χρησιμοποιήθηκε είναι γνωστή και ως προσέγγιση \textit{πεπερασμένων διαφορών} \cite{1995_Wilmott_BOOK}, και δίνεται από \cite{2021_Moffat_BOOK}: 

\begin{equation}\label{eq:autorss}
\begin{split}
\frac{\partial F}{\partial x_1} \delta x_1 &= \left(\lim_{\Delta x \to 0} \frac{F_{x_1 + \Delta x_1} - F_{x_1}}{\Delta x_1} + \lim_{\Delta x \to 0} \frac{F_{x_1} - F_{x_1 - \Delta x_1}}{\Delta x_1} \right) \frac{\delta x_1}{2}\\
&\approx \frac{\left(F_{x_1 + \Delta x_1} - F_{x_1} \right) + \left(F_{x_1} - F_{x_1 - \Delta x_1} \right)}{2}
\end{split}
\end{equation}

\noindent Ουσιαστικά λαμβάνεται ο μέσος όρος της \enquote{εμπρόσθιας} και \enquote{οπίσθιας} διαφοράς της συνάρτησης $F(x_1)$ με την απόλυτη τιμή της συνεισφοράς του σφάλματος $\delta x_i$. Πρέπει επίσης να σημειωθεί ότι η \prettyref{eq:autorss} ισχύει υπό την προϋπόθεση ότι το σχετικό σφάλμα είναι μικρό, δηλαδή ${\delta x_1 / F_{x1} \ll 1}$.

Τα βήματα που υιοθετούνται στον \prettyref{code:uncertaintyprop} συνοψίζονται στο \prettyref{fig:flowchart}. Κάθε μεταβλητή που χρησιμοποιείται στον προσδιορισμό της συνάρτησης $F$ είναι καταχωρημένη στο άνυσμα $x_i$, ενώ η αντίστοιχη αβεβαιότητά της στο άνυσμα $\delta x_i$. Με τις υπολογισθείσες ποσότητες των παραγώγων, συμπληρώνεται ο Ιακωβιανός πίνακας $J$ και, κατ' επέκταση, το πρόβλημα λύνεται σε μορφή πινάκων όπως περιγράφεται από τον \citeauthor{1998_Arras_TECH_REPORT} \cite{1998_Arras_TECH_REPORT}.

\begin{figure}[!htbp]
\centering

\tikzstyle{startstop} = [rectangle, rounded corners, minimum width=1.5cm, minimum height=1cm, align = center, draw=black, fill=lightgray!20]
\tikzstyle{junction} = [circle, draw=black, fill=white]
\tikzstyle{io} = [trapezium, trapezium left angle=70, trapezium right angle=110, minimum width=1.5cm, minimum height=1cm, text width = 1.5cm, align = center, draw=black, fill=lightgray!20]
\tikzstyle{process1} = [rectangle, minimum width=3cm, minimum height=1cm, align = center , draw=black, fill=lightgray!20]
\tikzstyle{process2} = [rectangle, minimum width=3cm, minimum height=1cm, align = center, draw=black, fill=lightgray!20]
\tikzstyle{decision} = [diamond, minimum width=3cm, minimum height=1cm, align = center, draw=black, fill=lightgray!20]
\tikzstyle{arrow1} = [thick,->,>=stealth]
\tikzstyle{arrow2} = [thick, dashed, ->, >=stealth]

\begin{adjustbox}{width=\textwidth,height=\textheight,keepaspectratio}
\begin{tikzpicture}[node distance=1.8cm]

\node (startm) [startstop] {Αρχή};
\node (in1) [io, below of=startm] {Διάβασε\\ $x_i, \delta x_i$};
\node (pro1) [process2, below of=in1] {Πήγαινε στην\\ υπορουτίνα\\ $F = f\left(x_1, \dots, x_N\right)$};
\node (pro2) [process1, below of=pro1] {$i = 1$\\$J = 0$};
\node (init) [text centered, left of=pro2, xshift=-2.5cm]{\textit{Αρχικοποίηση}};
\node (junc) [junction, below of=pro2, yshift=0.8cm] {};
\node (pro3a) [process1, below of=junc, yshift = 0.45cm] {Πήγαινε στην\\ υπρουτίνα\\ $\partial F / \partial x_i$};
\node (starts) [startstop, right of=pro3a, xshift=5cm] {Υπορουτίνα\\ $\partial F / \partial x_i$};
\node (pro1s) [process2, below of=starts, yshift=-0.5cm] {$F_{+ \delta_i} = f\left(x_1, \dots, x_N + \delta x_1, \dots, \delta x_N\right)$\\$F_{- \delta x_i} = f\left(x_1, \dots, x_N - \delta x_1, \dots, \delta x_N\right)$};
\node (pro2s) [process2, below of= pro1s] {$\displaystyle\frac{\partial F}{\partial x_i} = \frac{F_{+\delta x_i} - F_{-\delta x_i}}{2\delta x_i}$};
\node (stops) [startstop, below of=pro2s] {Επιστροφή};
\node (pro3b) [process1, below of=pro3a] {$J(i) =\displaystyle \frac{\partial F}{\partial x_i}$};
\node (pro3c) [process1, below of=pro3b] {$i = i + 1$};
\node (dec) [decision, below of=pro3c, yshift = -0.5cm] {$i < N$};
\node (pro4) [process1, below of=dec, yshift = -0.5cm] {$\delta F = \sqrt{\left(J \delta x\right) ^ 2}$};
\node (in2) [io, below of=pro4] {Τύπωσε\\ $F, \delta F$};
\node (stopm) [startstop, below of=in2] {Τέλος};

\draw [arrow1] (startm) -- (in1);
\draw [arrow1] (in1) -- (pro1);
\draw [arrow1] (pro1) -- (pro2);
\draw [thick] (pro2) -- (junc);
\draw [arrow1] (junc) -- (pro3a);
\draw [arrow1] (pro3a) -- (pro3b);
\draw [arrow1] (pro3b) -- (pro3c);
\draw [arrow1] (pro3c) -- (dec);
\draw [arrow1] (dec) -- node[right]{Όχι} (pro4);
\draw [arrow1] (pro4) -- (in2);
\draw [arrow1] (in2) -- (stopm);

\draw [very thick, ->, >=stealth, shorten >=6pt, -latex'] (init) -- (pro2);

% λούπα
\draw [arrow1] (dec) -- node[below]{Ναί} ++(-3cm,0) |- (junc);

\draw [arrow2] (pro3a) -- (starts);
\draw [arrow1] (starts) -- (pro1s);
\draw [arrow1] (pro1s) -- (pro2s);
\draw [arrow1] (pro2s) -- (stops);
\draw [arrow2] (stops) -- ++(-4cm,0) |- ([yshift=-8mm] pro3a);


\end{tikzpicture}
\end{adjustbox}
\caption{Διάγραμμα ροής για γραμμική διάδοση σφάλματος}\label{fig:flowchart}
\end{figure}

Το μόνο που απαιτείται λοιπόν στην ανάλυση δεδομένων είναι η δήλωση DRE (βλ. \prettyref{eq:DRE}) και οι τιμές των επιμέρους μεταβλητών $x_i$, συνοδευόμενες φυσικά από τις αντίστοιχες αβεβαιότητές τους $\delta x_i$. Λαμβάνοντας υπόψη το παράδειγμα από τη \prettyref{eq:Re}, η σχετική δήλωση σε κώδικα \matlab είναι

\begin{lstlisting}[firstnumber=238]
% Συνάρτηση υπολογισμού αριθμού Reynolds
Re = @(Q, density, dviscosity, dOuter, dInner) 4 * Q * density / (pi * dviscosity * (dOuter + dInner));
\end{lstlisting}

\noindent όπου η τιμή καθώς και η αβεβαιότητα του αριθμού Reynolds υπολογίζονται από

\begin{lstlisting}[firstnumber=264]
% Reynolds
[Reynolds(1, i, k), ReynoldsErr(1, i, k)] = UncertaintyPropagation(Re, [flowrate density dviscosity dOuter dInner], [uflowrate udensity udviscosity uDim uDim]);
\end{lstlisting}

\noindent Οι ανωτέρω γραμμές κώδικα ανήκουν στον \prettyref{code:datapadding}, ενώ η συνάρτηση Uncertainty Propagation αποτελεί τον \prettyref{code:uncertaintyprop}.


\section{Αξιολόγηση ποιότητας αποτελεσμάτων}

\noindent H \prettyref{eq:rss} δύναται να γραφτεί και σε αδιάστατη μορφή - μορφή ιδιαίτερα χρήσιμη στα πρώτα στάδια σχεδίασης ενός πειράματος - γνωστή και ως \textit{a priori} ανάλυση αβεβαιότητας \cite{1985_Moffat} ή \textit{pre-test} ανάλυση \cite{1973_Abernathy_TECH_REPORT}.

Στην ειδική περίπτωση που η \prettyref{eq:rss} δύναται να εκφραστεί ως γινόμενο των επιμέρους της μεταβλητών:

\begin{equation}\label{eq:prod}
F = C x_1^a x_2^b x_3^c x_4^d \dots
\end{equation}

\noindent τότε η αδιάστατη μορφή της θα είναι:

\begin{equation}\label{relrss}
\frac{\delta F}{F} = \left\{\left(a \frac{\delta x_1}{x_1} \right)^2 + \left(b \frac{\delta x_2}{x_2} \right)^2 + \left(c \frac{\delta x_3}{x_3} \right)^2 + \left(d \frac{\delta x_4}{x_4} \right)^2 + \dots \right\}^{1/2}
\end{equation}

\noindent στην οποία ο όρος $\delta F / F$ δηλώνεται ως σχετική αβεβαιότητα του αποτελέσματος, και οι παράγοντες $\delta x_i / x_i$ ως σχετικές αβεβαιότητες των μεταβλητών. Οι σταθερές δύναμης $(a,b, \text{κλπ})$ που πολλαπλασιάζονται με τις σχετικές αβεβαιότητες των μεταβλητών, ορίζονται ως συντελεστές μεγέθυνσης αβεβαιότητας \cite{2018_HughW.Coleman_BOOK}, και δηλώνουν τη συμβολή της αντίστοιχης αβεβαιότητας της εκάστοτε μεταβλητής, $\delta x_i$, στην αβεβαιότητα του αποτελέσματος, $\delta F$. Στη \prettyref{eq:prod} για παράδειγμα, μια τιμή $a > 1$ υποδεικνύει ότι η επιρροή της αβεβαιότητας $\delta x_i$, μεγεθύνεται όσο αυτή διαδίδεται μέσω της εξίσωσης συσχέτισης (DRE).

Για την αξιολόγηση της ποιότητας των μετρήσεων που έλαβαν χώρα στο συγκεκριμένο πείραμα, θα χρησιμοποιηθεί η ανάλυση αβεβαιότητας με το πέρας της ανάλυσης δεδομένων. Αυτό θα δώσει μια ξεκάθαρη εικόνα για τα αναμενόμενα εύρη σχετικών σφαλμάτων, και θα καταστήσει την αξιολόγηση των αντίστοιχων μεταβλητών τους πιο αντικειμενική.

Οι μεταβλητές που θα εξεταστούν θα είναι (i) η παροχή, $\dot{\volume}$, (ii) η ηλεκτρική ισχύς, $\dot{W}$, (iii) ο αριθμός Reynolds, $Re$, (iv) ο τοπικός αριθμός Nusselt, $Nu_x$, και (v) ο ροή θερμότητας του συστήματος προς τον αέρα, $\dot{Q}_{sys.}$.

Η συμβολή που θα έχουν οι επιμέρους αβεβαιότητες στα αντίστοιχα τελικά αποτελέσματα θα εξαρτηθεί από τις τιμές των μεταβλητών τους, ορισμένες από τις οποίες θα μεταβάλλονται κατά τη διάρκεια των πειραμάτων. Για το λόγο αυτό θα χρησιμοποιηθούν τα δεδομένα και οι μετρήσεις όλων των πραγματοποιηθέντων πειραμάτων.
 
Τέλος, πρέπει να σημειωθεί ότι οι αβεβαιότητες που σχετίζονται με τις θερμοδυναμικές ιδιότητες του αέρα $\left(\rho, \, \kappa, \mu \right)$ θεωρήθηκαν μηδενικές. Πράγματι, οι ιδιότητες αυτές υπολογίστηκαν βάσει της μέσης θερμοκρασίας εισόδου και εξόδου του αέρα σε συνδυασμό με τις εκάστοτε εμπειρικές τους σχέσεις, έχοντας αβεβαιότητες $\delta T_{\infty} = \qty{0.01}{\degreeCelsius}$ και $\delta T_{\text{εξ.}} = \qty{0.02}{\degreeCelsius}$ αντίστοιχα. Συνδυάζοντας αμφότερες για την αβεβαιότητα μέσης τιμής αυτών βρίσκουμε $\delta T_{avg.} = \qty{0.0158}{\degreeCelsius}$, ενώ το τυπικό σφάλμα ενός θερμοστοιχείου κυμαίνεται στο $\qty{0.18}{\degreeCelsius}$ - τάξη μεγέθους μεγαλύτερο.

\subsection{Σχετικό σφάλμα ογκομετρικής παροχής}\label{flowunc}

\noindent Εφαρμόζοντας την \ref{relrss} στην \prettyref{eq:floww} έχουμε:

\begin{equation*}
\frac{\delta \dot{\volume}}{\dot{\volume}} = \left\{\left(-1\frac{\delta t}{t}\right) ^ 2\right\} ^ {1/2}
\end{equation*}

\noindent όπου η συνολική αβεβαιότητα μέτρησης χρόνου είναι:

\begin{equation*}
\delta t = \qty{1.86}{\second}
\end{equation*}

\noindent H αβεβαιότητα της ογκομετρικής παροχής καθορίζεται εξ' ολοκλήρου από μία μεταβλητή, αυτή του χρόνου. Οπότε, η συνολική αβεβαιότητα θα εξαρτηθεί από τις τιμές του χρόνου, $t$, ο οποίος μεταβάλλεται για διάφορες τιμές της παροχής, $\dot{\volume}$. Το μόνο που χρειάζεται λοιπόν είναι τα αντίστοιχα εύρη τιμών $\dot{\volume}, \, t$, όπου υπολογίζεται ο λόγος $\delta t / t$ και εν συνεχεία δημιουργείται γράφημα για της αντίστοιχες τιμές $\dot{\volume}$.

Οι πειραματικές τιμές ογκομετρικής παροχής κυμαίνονται από \qty{8.18d-4}{\metre\cubed\per\second} μέχρι \qty{16d-4}{\metre\cubed\per\second}, και αντιστοιχούν σε \qty{122,28}{\second} με \qty{62,15}{\second} μετρήσεις χρόνου. Τα αποτελέσματα φαίνονται στο \prettyref{plt:relflowE}.

\begin{figure}[tbp]
\centering
\subimport{Chapter4/Figs/pdftex}{relerrQ.pdf_tex}
\caption{Σχετικό σφάλμα ογκομετρικής παροχής για εύρος λειτουργίας.}\label{plt:relflowE}
\end{figure}

Η σχετική αβεβαιότητα της $\dot{\volume}$ λαμβάνει μέγιστη τιμή για υψηλές τιμές της $\dot{\volume}$, όπου η $t$ έχει μικρές τιμές. Γενικά το σφάλμα κυμαίνεται από \qty{1.5}{\percent} μέχρι \qty{3}{\percent}, θεμιτό αποτέλεσμα στο πλαίσιο της παρούσας εργασίας.

\subsection{Σχετικό σφάλμα ηλεκτρικής ισχύος}\label{powerunc}

\noindent Κατ' αντιστοιχία, εφαρμόζοντας την \ref{relrss} στην \prettyref{eq:fanpower} έχουμε:

\begin{equation*}
\frac{\delta \dot{W}}{\dot{W}} = \left\{\left(\frac{\delta V}{V} \right) ^ 2 + \left(\frac{\delta I}{I} \right) ^ 2 \right\} ^ {1/2} 
\end{equation*}

\noindent όπου οι συνολικές αβεβαιότητες των μετρήσεων τάσης και ρεύματος είναι:

\begin{align*}
\delta V &= \qty{0.22}{\volt}\\
\delta I &= \qty{0.03}{\ampere}
\end{align*}

\noindent Η αβεβαιότητα της ηλεκτρική ισχύος καθορίζεται από δύο μεταβλητές, αυτές της τάσης και του ρεύματος. Οπότε η συνολική αβεβαιότητα θα εξαρτηθεί από τις τιμές του ρεύματος, $V$, σε συνδυασμό με τις τιμές του ρεύματος, $I$, που μεταβάλλονται για διάφορες τιμές της ηλεκτρικής ισχύος, $\dot{W}$. Το μόνο που χρειάζεται λοιπόν είναι τα αντίστοιχα εύρη τιμών $V-I, \dot{W}$, όπου υπολογίζονται οι λόγοι $\delta V / V$ και $\delta I / I$ και, εν συνεχεία, δημιουργείται γράφημα για αντίστοιχες τιμές $\dot{W}$.

Τα πειραματικά ζεύγη τιμών τάσης-ρεύματος κυμαίνονται από \qty{3.96}{\volt} - \qty{0.13}{\ampere} μέχρι \qty{26.97}{\volt} - \qty{1.7}{\ampere}, και αντιστοιχούν σε \qty{0.51}{\watt} με \qty{37.16}{\watt} μετρήσεις ηλεκτρικής ισχύος. Τα αποτελέσματα φαίνονται στο \prettyref{plt:relpowerE}.

\begin{figure}[!htbp]
\centering
\subimport{Chapter4/Figs/pdftex}{relerrP.pdf_tex}
\caption{Σχετικό σφάλμα ισχύος για εύρος λειτουργίας.}\label{plt:relpowerE}
\end{figure}

Η σχετική αβεβαιότητα της $\dot{W}$ έχει μέγιστη τιμή για χαμηλές τιμές της $\dot{W}$, όπου οι $I, \, V$ έχουν μικρές τιμές. Το σφάλμα μικραίνει αισθητά από \qty{10}{\watt} και μετά, ακολουθώντας φθίνουσα πορεία. Συμπεραίνουμε ότι το συγκεκριμένο μετρητικό δεν είναι κατάλληλο για χαμηλές μετρήσεις ισχύος, πράγμα που θα φανεί σε περαιτέρω επεξεργασία δεδομένων στο \prettyref{ch:results}.

\subsection{Σχετικό σφάλμα Reynolds}\label{reynoldsunc}

\noindent Εφαρμόζοντας την \ref{relrss} στην \prettyref{eq:reynolds}, και λαμβάνοντας υπόψη την \prettyref{eq:floww}, έχουμε:

\begin{equation*}
\frac{\delta Re}{Re} = \left\{\left(\frac{\delta \rho}{\rho}\right)^2 + \left(-1\frac{\delta \mu}{\mu}\right)^2 + \left(\frac{\delta t}{t}\right)^2 + \left(\frac{\delta \Sigma D}{\Sigma D}\right)^2\right\}^{1/2}
\end{equation*}

\noindent όπου οι συνολικές αβεβαιότητες των μετρήσεων πυκνότητας, δυναμικού ιξώδους και χρόνου είναι:

\begin{align*}
\delta \rho &= \qty{0}{\kilogram\per\meter\cubed}\\
\delta \mu &= \qty{0}{\kilogram\per\meter\per\second}\\
\delta t &= \qty{1.86}{\second}\\
\end{align*}

\noindent ενώ η αβεβαιότητα του αθροίσματος των εσωτερικού και εξωτερικού κυλίνδρων λαμβάνεται από:

\begin{align*}
\delta \Sigma D &= \delta \left(D_{\text{εξ.}} + D_{\text{εσ.}}\right) = \left\{\left(\delta D_{\text{εξ.}}\right) ^ 2 + \left(\delta D_{\text{εσ.}}\right)^2 \right\} ^ {1/2}\\
&= \left\{\left(\num{0.05d-3}\right) ^ 2 + \left(\num{0.05d-3}\right) ^ 2 \right\} ^ {1/2} = \qty{0.71e-3}{\metre} \\
\end{align*}

\noindent Η αβεβαιότητα του αριθμού Reynolds καθορίζεται από δύο μεταβλητές, αυτές του χρόνου και υδραυλικής διαμέτρου της δακτυλιοειδούς διάταξης. Οι λοιπές μεταβλητές είναι μηδέν λόγω παραδοχών που έγιναν στην ανάλυση δεδομένων. Η συνολική αβεβαιότητα θα εξαρτηθεί άρα, από τις τιμές χρόνου, $t$, σε συνδυασμό με την τιμή του αθροίσματος των διαμέτρων, $\Sigma D$, για διάφορες τιμές του αριθμού Reynolds, $Re$. Το μόνο που χρειάζεται λοιπόν είναι τα εύρη τιμών $t$ και η σχετική αβεβαιότητα $\delta \Sigma D$, βάσει των οποίων υπολογίζονται οι όροι $\delta t / t$ και $\delta \Sigma D / \Sigma D$, και εν συνεχεία δημιουργείται γράφημα για διάφορες τιμές του $Re$.

Το άθροισμα των διαμέτρων ισούται με \qty{62e-3}{\metre} με σχετική αβεβαιότητα \qty{0.0011}{\percent}, μια παραπάνω από ικανοποιητική τιμή. Οι πειραματικές μετρήσεις χρόνου κυμαίνονται από \qty{62.15}{\second} μέχρι \qty{122.28}{\second}. Τα αποτελέσματα φαίνονται στο \prettyref{plt:relreE}.

\begin{figure}[!htbp]
\centering
\subimport{Chapter4/Figs/pdftex}{relerrRe.pdf_tex}
\caption{Σχετικό σφάλμα Reynolds για εύρος λειτουργίας.}\label{plt:relreE}
\end{figure}

Όπως στη περίπτωση της $\dot{\volume}$ (βλ. \prettyref{plt:relflowE}), έτσι κι εδώ, η αβεβαιότητα της $Re$ λαμβάνει μέγιστη τιμή για υψηλές τιμές της $Re$, όπου ο $t$ έχει μικρές τιμές. Αυτό είναι λογικό καθώς σε αμφότερες περιπτώσεις η μεταβλητή που παίζει καθοριστικό ρόλο στη συνολική αβεβαιότητα είναι ο $t$. Η ύπαρξη της $\Sigma D$, εντός τετραγωνικής ρίζας, καθιστά τη μορφή της δεύτερης μη γραμμική. Η μέγιστη τιμή αβεβαιότητας της $Re$ είναι της τάξεως του \qty{4}{\percent}, γεγονός που, όπως και στην περίπτωση της $\dot{\volume}$, καθιστά θεμιτή την διαδικασία μέτρησης στο πλαίσιο της παρούσας εργασίας.

\subsection{Σχετικό σφάλμα τοπικού Nusselt}\label{nusseltunc}

\noindent Εφαρμόζοντας την \ref{relrss} στην \prettyref{eq:localnusselt}, και λαμβάνοντας υπόψη τη \prettyref{eq:localhtc}, έχουμε:


\begin{align*}
\frac{\delta Nu_x}{Nu_x} = &\left\{ \left(-1\frac{\delta L}{L}\right) ^ 2 + \left(-1\frac{\delta D_{\text{εσ.}}}{D_{\text{εσ.}}}\right) ^ 2 + \left(\frac{\delta D_h}{D_h}\right) ^ 2 \right.\\
 &\left. \quad + \left(\frac{\delta W}{W}\right) ^ 2 + \left(\frac{\delta \Delta T_x}{\Delta T_x}\right) ^ 2 + \left(-1\frac{\delta k_{\text{αέρα}}}{k_{\text{αέρα}}}\right) ^ 2 \right\} ^ {1/2}
\end{align*}

\noindent όπου οι συνολικές αβεβαιότητες των μετρήσεων μήκους/ακτίνας εσωτερικού κυλίνδρου, ισχύος της αντίστασης και θερμικής αγωγιμότητας αέρα είναι:

\begin{align*}
\delta L &= \qty{0.05e-3}{\metre}\\
\delta D_{\text{εσ.}} &= \qty{0.05e-3}{\metre}\\
\delta W &= \qty{0.50}{\watt}\\
\delta k_{\text{αέρα}} &= \qty{0}{\watt\per\metre\per\kelvin} 
\end{align*}

\noindent ενώ η συνολική αβεβαιότητα της υδραυλικής διαμέτρου λαμβάνεται από:

\begin{align*}
\delta D_h &= \delta \left(D_{\text{εξ.}} - D_{\text{εσ.}}\right) = \left\{\left(\delta D_{\text{εξ.}}\right) ^ 2 + \left(\delta D_{\text{εσ.}}\right)^2 \right\} ^ {1/2}\\
&= \left\{\left(\num{0.05d-3}\right) ^ 2 + \left(\num{0.05d-3}\right) ^ 2 \right\} ^ {1/2} = \qty{0.71e-3}{\metre}\\
\end{align*}

\noindent και η συνολική αβεβαιότητα θερμοκρασιακής διαφοράς αντίστασης-αέρα:

\begin{align*}
\delta \Delta T &= \delta \left(T_{\text{αντ., x}} - Τ_{\text{αέρα, x}}\right) = \left\{\left(\delta T_{\text{αντ., x}}\right) ^ 2 + \left(\delta Τ_{\text{αέρα, x}}\right)^2 \right\} ^ {1/2}\\
\end{align*}

\noindent όπου ο δείκτης x υποδηλώνει την επί της αντίστασης θέση του εκάστοτε θερμοστοιχείου.

Επίσης, οι αβεβαιότητες στη θερμοκρασιακή διαφορά αντίστασης-αέρα εξαρτώνται από τις αβεβαιότητες των σχετικών θερμοστοιχείων και τις αβεβαιότητες παρεμβολής αέρα\footnote{είναι της τάξεως του δεκάτου του χιλιοστού οπότε η συνεισφορά τους είναι ανεπαίσθητη, αναφέρονται κυρίως για λόγους πληρότητας}, για κάθε διάταξη. Οι τελευταίες υπολογίζονταν κάθε φόρα χρησιμοποιώντας τον \prettyref{code:lsqfit}, ενώ αυτές των θερμοστοιχείων είναι:

\begin{align*}
\delta T_{\text{αντ., 1}} &= \qty{0.14}{\degreeCelsius}, & \delta T_{\text{αντ., 2}} &= \qty{0.17}{\degreeCelsius}, & \delta T_{\text{αντ., 3}} &= \qty{0.18}{\degreeCelsius}\\
\delta T_{\text{αντ., 4}} &= \qty{0.18}{\degreeCelsius}, & \delta T_{\text{αντ., 5}} &= \qty{0.18}{\degreeCelsius}, & \delta T_{\text{αντ., 6}} &= \qty{0.18}{\degreeCelsius}\\
\delta T_{\text{αντ., 7}} &= \qty{0.17}{\degreeCelsius}, & \delta T_{\text{αντ., 8}} &= \qty{0.16}{\degreeCelsius}, & \delta T_{\text{αντ., 9}} &= \qty{0.14}{\degreeCelsius}
\end{align*}

\noindent Η αβεβαιότητα του τοπικού αριθμού Nusselt καθορίζεται από πέντε μεταβλητές, αυτές του μήκους και της ακτίνας του εσωτερικού κυλίνδρου, της ισχύος της αντίστασης, της υδραυλικής διαμέτρου του δακτυλιοειδή σωλήνα και αυτή της θερμοκρασιακής διαφοράς αντίστασης-αέρα. Η αβεβαιότητα της θερμικής αγωγιμότητας του αέρα είναι μηδέν λόγω παραδοχών που έγιναν στην ανάλυση δεδομένων.

Η συνολική αβεβαιότητα θα εξαρτηθεί άρα από την τιμή του μήκους αντίστασης, $L$, την τιμή της εσωτερικής διαμέτρου του κυλίνδρου, $D_{\text{εσ.}}$, τη τιμής της ισχύος της αντίστασης, $\dot{W}$, τη τιμή της υδραυλικής διαμέτρου, $D_h$, και τις τιμές των θερμοκρασιακών διαφορών αντίστασης-αέρα, $\delta \Delta T_x$, για διάφορες τιμές του τοπικού αριθμού Nusselt, $Nu_x$.

Οι τέσσερις πρώτες των μεταβλητών είναι σταθερές για όλες τις πειραματικές διατάξεις, δηλαδή θα ισχύουν $L = \qty{0.9}{\metre}, \, D_{\text{εσ.}} = \qty{22d-3}{\metre}, \, \dot{W} = \qty{18.58}{\watt}\, \text{και}\, D_h = \qty{18d-3}{\metre}$ για όλα τα πειράματα που έλαβαν χώρα. Το μόνο που χρειάζεται λοιπόν είναι τα εύρη τιμών $\delta \Delta T_x$, για κάθε θερμοστοιχείο, βάσει των οποίων υπολογίζεται ο όρος  $\delta \Delta T_x / \Delta T_x$, και εν συνεχεία δημιουργείται γράφημα για τις διάφορες τιμές του $Nu_x$.

Τα εύρη των τιμών που λαμβάνει κάθε στοιχείο είναι (βλ. \prettyref{app:measurements}):

\sisetup{range-phrase =-, range-units=single}

\begin{align*}
T_{\text{αντ., 1}} &= \SIrange{24.57}{55.91}{\degreeCelsius} & T_{\text{αντ., 2}} &= \SIrange{28.59}{53.23}{\degreeCelsius} & T_{\text{αντ., 3}} &= \SIrange{33.10}{52.48}{\degreeCelsius}\\
T_{\text{αντ., 4}} &= \SIrange{36.40}{50.97}{\degreeCelsius} & T_{\text{αντ., 5}} &= \SIrange{38.14}{49.15}{\degreeCelsius} & T_{\text{αντ., 6}} &= \SIrange{38.46}{51.47}{\degreeCelsius}\\
T_{\text{αντ., 7}} &= \SIrange{36.16}{52.90}{\degreeCelsius} & T_{\text{αντ., 8}} &= \SIrange{34.30}{53.45}{\degreeCelsius} & T_{\text{αντ., 9}} &= \SIrange{33.68}{55.12}{\degreeCelsius}
\end{align*}

\noindent ενώ οι τοπικοί Nusselt κυμαίνονται, 

\begin{align*}
Nu_1 &= 59 - 340 & Nu_2 &= 68 - 557 & Nu_3 &= 75 - 283\\
Nu_4 &= 86 - 229 & Nu_5 &= 98 - 207 & Nu_6 &= 93 - 223\\
Nu_7 &= 94 - 369 & Nu_8 &= 100 - 769 & Nu_9 &= 99 - 160
\end{align*}

\noindent Τα αποτελέσματα φαίνονται στο \prettyref{plt:relTE}.

\begin{figure}[!htbp]
\centering
\subimport{Chapter4/Figs/pdftex}{relerrNu.pdf_tex}
\caption{Σχετικό σφάλμα Nusselt για εύρος λειτουργίας.}\label{plt:relTE}
\end{figure}

Η σχετική αβεβαιότητα του $Nu_x$, για όλα τα θερμοστοιχεία, είναι μεγαλύτερη για υψηλές τιμές του $Nu_x$, όπου η αβεβαιότητα $\delta \Delta T_x$ επικρατεί των υπολοίπων. Χαμηλές τιμές της $\Delta T_x$ φαίνεται να επισύρουν το σχετικό σφάλμα. Σε ορισμένες περιπτώσεις αυτό καθίσταται αρκετά αισθητό, όπως στα θερμοστοιχεία 1,2 και 9 όπου παρουσιάστηκε μέγιστο σχετικό σφάλμα της τάξεως του \qty{8}{\percent}, ενώ στα λοιπά κυμαινόταν από \qty{5.5}{\percent} μέχρι \qty{6.5}{\percent}, για τα υπό εξέταση εύρη τιμών. Συμπεραίνουμε λοιπόν ότι τα θερμοστοιχεία δεν είναι κατάλληλα για μικρές τιμές μέτρησης θερμοκρασίας, και ότι προβλέπεται επαναξιολόγηση (ή στην εσχάτη των περιπτώσεων, αλλαγή) ως προς την καταλληλότητά τους. Για πρακτικούς λόγους, ωστόσο, θα συνεχιστεί κανονικά η εργασία αναμένοντας έντονα σφάλματα στα τελικά αποτελέσματα.

\subsection{Σχετικό σφάλμα ροής θερμότητας προς τον αέρα}

\noindent Για τελευταία φορά (κούρασα, ξέρω), εφαρμόζοντας την \ref{relrss} στην \prettyref{eq:htrate} έχουμε:

\begin{equation*}
\frac{\delta \dot{Q}_{sys.}}{\dot{Q}_{sys.}} = \left\{\left(\frac{\delta \rho}{\rho}\right)^2 + \left(\frac{\delta t}{t}\right)^2 + \left(\frac{\delta C_p}{C_p}\right)^2 + \left(\frac{\delta \Delta T}{\Delta T}\right)^2 \right\} ^ {1/2}
\end{equation*}

\noindent όπου οι συνολικές αβεβαιότητες των μετρήσεων πυκνότητας, χρόνου και ειδικής θερμότητας είναι:

\begin{align*}
\delta \rho &= \qty{0}{\kilo\gram\per\cubed\metre}\\
\delta t &= \qty{1.86}{\second}\\
\delta C_p &= \qty{0}{\joule\per\kilogram\per\kelvin}\\
\end{align*}

\noindent ενώ η συνολική αβεβαιότητα της διαφοράς θερμοκρασιών εξόδου-εισόδου λαμβάνεται από:

\begin{align*}
\delta \Delta T &= \delta \left(T_{\infty} - Τ_{\text{εξ.}}\right) = \left\{\left(\delta T_{\infty}\right) ^ 2 + \left(\delta Τ_{\text{εξ.}}\right)^2 \right\} ^ {1/2}\\
 &= \left\{0.01 ^ 2 + 0.04 ^ 2\right\} ^ {1/2} = \qty{0.04}{\degreeCelsius}
\end{align*}

\noindent Η αβεβαιότητα του ρυθμού ροής θερμότητας καθορίζεται από δύο μεταβλητές, αυτές του χρόνου και της θερμοκρασιακής διαφοράς. Οι υπόλοιπες αβεβαιότητες είναι μηδέν λόγω παραδοχών που έγιναν στην ανάλυση δεδομένων. Η συνολική αβεβαιότητα θα εξαρτηθεί από τις τιμές του χρόνου, $t$, σε συνδυασμό με τις τιμές της θερμοκρασιακής διαφοράς, $\Delta T$, που μεταβάλλονται για διάφορες τιμές του ρυθμού μετάδοσης θερμότητας, $\dot{Q}_{sys.}$. Το μόνο που χρειάζεται λοιπόν είναι τα ανάλογα εύρη τιμών $t, \, \Delta T$ όπου υπολογίζονται οι λόγοι $\delta t / t$ και $\delta \Delta T / T$ και, εν συνεχεία, δημιουργείται γράφημα για διάφορες τιμές $\dot{Q}_{sys.}$.

Οι πειραματικές μετρήσεις θερμοκρασίας κυμαίνονται από \qty{7,6}{\degreeCelsius} μέχρι \qty{16.34}{\degreeCelsius}, ενώ οι μετρήσεις χρόνου από \qty{62.15}{\second} μέχρι \qty{122.28}{\second}. Για αποφυγή υπερφόρτωσης του γραφήματος, έχουν ληφθεί μονάχα οι δύο ακραίες τιμές της θερμοκρασιακής διαφοράς για το εύρος των μετρήσεων χρόνου. Τα αποτελέσματα φαίνονται στο \prettyref{plt:htrateE}.

\begin{figure}[!htbp]
\centering
\subimport{Chapter4/Figs/pdftex}{relerrQconv.pdf_tex}
\caption{Σχετικό σφάλμα ρυθμού μετάδοσης θερμότητας για εύρος λειτουργίας.}\label{plt:htrateE}
\end{figure}

Η συνεισφορά της $\Delta T$ μοιάζει ανεπαίσθητη. Η σχετική αβεβαιότητα του $\dot{Q}_{sys.}$ είναι μεγαλύτερη για υψηλές τιμές του $\dot{Q}_{sys.}$, όπου η αβεβαιότητα $\delta t$ επικρατεί αυτής της $\Delta T$ . Οι τιμές της $\delta \dot{Q}_{sys.}$ κυμαίνονται από \qty{3.2}{\percent} μέχρι \qty{4.1}{\percent}, για όλο τα πιθανά εύρη πειραματικών τιμών, γεγονός που καθιστά τον υπολογισμό της αρκετά ικανοποιητικό στο πλαίσιο του συγκεκριμένου πειράματος. 