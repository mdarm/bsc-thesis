%!TEX root = ../thesis.tex
% ******************************* Thesis Appendix C ****************************
\chapter{Κώδικας \matlab}\label{app:codeAppendix}
\begin{refsection}

\noindent Στο \prettyref{app:codeAppendix} συγκεντρώνονται οι αλγόριθμοι που χρησιμοποιήθηκαν για την ανάλυση δεδομένων, τα αποτελέσματα της οποίας βρίσκονται στη σελ.~\pageref{results} υπό την ετικέτα \texttt{results.txt}. Οι καταγεγραμμένοι αλγόριθμοι συντάχθηκαν από τον συγγραφέα, ενώ αλγόριθμοι τρίτων προσώπων, που ενίοτε χρησιμοποιούνταν στην ανάλυση, αναφέρονται στις παραπομπές.

Ιδιαίτερα χρήσιμος ήταν ο κώδικας του \citeauthor{Kis2021} \cite{Kis2021} (βλ. \prettyref{code:lsqfit}), ο οποίος προεκτάθηκε ούτως ώστε να συμπεριλαμβάνει αβεβαιότητες, και η μεθοδολογία του οποίου αναλύεται διεξοδικά στο \prettyref{app:lsqAppendix}. Γλύτωσε από το συγγραφέα αρκετές ώρες εργασίας.

Χρήσιμα ήταν επίσης η συνάρτηση αριθμητικής ολοκλήρωσης\footnote{\href{https://se.mathworks.com/help/matlab/ref/trapz.html}{trapz}} \cite{matlabtrapz} και η στατιστική εργαλειοθήκη\footnote{\href{https://www.mathworks.com/products/curvefitting.html}{Curve Fitting toolbox}} \cite{matlabcurvefitting} του \matlab. Αμφότερα συνέβαλαν στην απλοποίηση ορισμένων, κατά τα άλλα, αρκετά περίπλοκων υπολογισμών (βλ. υποενότητες \ref{reppower} και \ref{repnu}).

Βιβλία που βοήθησαν, και ενέπνευσαν, στην υλοποίηση του κώδικα είναι της \citeauthor{2018_Attaway_BOOK} \cite{2018_Attaway_BOOK}, του \citeauthor{2010_Johnson_BOOK} \cite{2010_Johnson_BOOK} και των \citeauthor{2012_Patera_BOOK} \cite{2012_Patera_BOOK}. Ενώ για τη σύνθεση γραφημάτων, καθοριστικό ρόλο έπαιξε η συνάρτηση του \citeauthor{Jong2016} \cite{Jong2016} σε συνδυασμό με το λογισμικό ανυσματικών γραφικών \inkscape \parencites{2009_Kirsanov_BOOK}{2017_Mark_BOOK} καθώς επίσης και το βιβλίο του \citeauthor{2001_Tufte_BOOK} \cite{2001_Tufte_BOOK}. Έγινε επίσης εκτενής χρήση της βιβλιοθήκης παραγωγής γραφημάτων του \LaTeX, \TikZ \cite{Tantau2013}.


\lstinputlisting[caption = {Προσδιορισμός διαστάσεων γραφημάτων}\label{code:plotdim}]{"Appendix3/matlab/PlotDimensions.m"}

\lstinputlisting[caption = {Αλλαγή διερμηνευτή}\label{code:changeint}]{"Appendix3/matlab/ChangeInterpreter.m"}

\lstinputlisting[caption = {Εκτίμηση αβεβαιότητας μετρητικών οργάνων}\label{code:measurementunc}]{"Appendix3/matlab/InstrumentUncertainty.m"}

\lstinputlisting[caption = {Διάδοση σφαλμάτων}\label{code:uncertaintyprop}]{"Appendix3/matlab/UncertaintyPropagation.m"}

\lstinputlisting[caption = {Σταθμισμένα ελάχιστα τετράγωνα}\label{code:lsqfit}]{"Appendix3/matlab/LeastSquaresFit.m"}

\lstinputlisting[caption = {Σταθμισμένος μέσος και διασπορά}\label{code:wvar}]{"Appendix3/matlab/WeightedVariance.m"}

\lstinputlisting[caption = {Βαθμονόμηση ογκομετρητή παροχής}\label{code:flowcallibration}]{"Appendix3/matlab/FlowCallibration.m"}

\lstinputlisting[caption = {Ανάλυση δεδομένων}\label{code:datapadding}]{"Appendix3/matlab/EgregiousDataPadding.m"}

%\definecolor{backcolour}{rgb}{0.95,0.95,0.92}
%\lstinputlisting[title = \fbox{\color{Black}test.dat}, backgroundcolor=\color{backcolour}, language ={}, frame=tb, numbers=none, basicstyle=\ttfamily, breaklines=true]{"Appendix3/matlab/results.txt"}
\vspace{25pt}
%\begin{landscape}
\VerbatimInput[reflabel={results}, framerule=0.6mm, fontfamily=courier]{Appendix3/matlab/results.txt}
%\end{landscape}

\printbibheading
\begin{english}
\printbibliography[heading=subbibliography, type = book, title = {Βιβλία}]
\printbibliography[heading=subbibliography, type = manual, title = {Εγχειρίδια}]
\printbibliography[heading=subbibliography, type = misc, title = {Ξένος κώδικας}]
\end{english}

\end{refsection}