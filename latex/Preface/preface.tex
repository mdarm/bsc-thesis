% ************************** Thesis Acknowledgements **************************
\definecolor{buff}{rgb}{0.94, 0.86, 0.51}
\nomenclature[z-TAÜ]{TAÜ}{Türk-Alman Üniversitesi}
\nomenclature[z-ΠΑ.Δ.Α.]{ΠΑ.Δ.Α.}{Πανεπιστήμιο Δυτικής Αττικής}
\nomenclature[z-Ε.ΔΙ.Π.]{Ε.ΔΙ.Π.}{Εργαστηριακό Διδακτικό Προσωπικό}

\begin{preface}

\noindent Η ένταξή μου στο Εργαστήριο Θερμοδυναμικής του \ITU\,το 2019, με σκοπό την πραγματοποίηση πρακτικής άσκησης, επέδρασε πολύ περισσότερο στη ζωή μου από το αρχικά αναμενόμενο. Οι δώδεκα αυτοί μήνες ήταν μήνες δημιουργικής έρευνας, η οποία τελικά οδήγησε, μεταξύ άλλων, στην παρούσα πτυχιακή εργασία. Στους μήνες αυτούς προσπάθησα να αποκομίσω όσο το δυνατόν περισσότερα - και να οικειοποιηθώ της περιοχής των θερμορευστών, ειδικότερα του τομέα των περιδινούμενων ροών - και θέλω να ελπίζω ότι η δουλειά που προέκυψε από την έρευνα αυτή, καθώς και η παρούσα εργασία έπεται να εξελιχθούν σε βαθμό τέτοιο ώστε να συμβάλλουν στην πρόοδο του τομέα που πραγματεύονται.

Ξεκινώντας θα ήθελα να ευχαριστήσω τον επιβλέποντα της πρακτικής μου άσκησης, καθηγητή του \ITU\,κ. Murat Çakan. Με τη δική του προτροπή άρχισε η ενασχόληση με το συγκεκριμένο θέμα, το οποίο αποδείχτηκε εξαιρετικά ενδιαφέρον για εμένα. Οι γνώσεις, οι εμπειρίες, οι μπακλαβάδες (δεν κρατήθηκα), η διάθεση και το ενδιαφέρον του οδήγησαν τη συγκεκριμένη δουλειά σε αυτό το σημείο, ενώ οποιαδήποτε στιγμή ήταν έτοιμος να προσφέρει ό,τι ήταν δυνατό ώστε να ξεπεραστεί κάθε δυσκολία.

Επίσης, στη μέχρι τώρα πορεία μου, αξιοσημείωτα είναι η προσοχή και το ενδιαφέρον που επέδειξε ο επιβλέπων της πτυχιακής μου εργασίας, και εισηγητής της πρακτικής μου άσκησης, καθηγητής του ΠΑ.Δ.Α. κ. Κωνσταντίνος-Στέφανος Νίκας, καθώς και το γεγονός ότι παραστάθηκε με εξαιρετική διάθεση, όποτε του το ζήτησα, σε οποιοδήποτε πρόβλημά μου. Τον ευχαριστώ ιδιαίτερα. Θερμές ευχαριστίες οφείλω και στα άλλα μέλη της τριμελούς επιτροπής, αναπληρωτή καθηγητή του ΠΑ.Δ.Α. κ. Κωνσταντίνο Μουστρή και Ε.ΔΙ.Π. του ΠΑ.Δ.Α. κ. Ιωάννη Σιγάλα.

Ευχαριστίες οφείλω και στον αγαπημένο φίλο και διδάκτορα του TAÜ, κ. Sefer Arda Serbes για την πολύτιμη συνεργασία του στην έρευνά μου, αλλά και στα μέλη του Εργαστηρίου Θερμοδυναμικής και ειδικότερα σε αυτούς που συνεργαστήκαμε στενότερα, συνδυάζοντας την έρευνά μας, με σκοπό να επιτευχθούν τα καλύτερα δυνατά αποτελέσματα. Για τη συνεργασία σε επιστημονική εργασία συναφούς αντικειμένου, θα ήθελα να ευχαριστήσω τον Anil Berk Atalar. \newpage
\thispagestyle{plain}

Ιδιαίτερη μνεία θα ήθελα να κάνω στον πρώην καθηγητή του ΠΑ.Δ.Α. κ. Κωνσταντίνο Γιαννακόπουλο, που χάθηκε πρόωρα. Ήταν από εκείνους τους ανθρώπους που στάθηκαν εμπνευστές στο δρόμο της ουσιαστικής γνώσης.

Τέλος θα ήθελα να ευχαριστήσω την οικογένειά μου, καθώς και τους φίλους μου για τη συμμετοχή και τη συμβολή τους σε όλες τις στιγμές της ζωής μου.

%\noindent Το παρόν έγγραφο συντάχθηκε με
%\begin{tcolorbox}[colback=buff]
%This is XeTeX, Version 3.141592653-2.6-0.999993 (MiKTeX 21.12.10) (preloaded format=xelatex 2021.12.18)  21 DEC 2021 15:00
%\end{tcolorbox}

\end{preface}