% ******************************************************************************
% ****************************** Custom Margin *********************************

% Add `custommargin' in the document class options to use this section
% Set {innerside margin / outerside margin / topmargin / bottom margin}  and
% other page dimensions
\ifsetCustomMargin
  \RequirePackage[left=37mm,right=30mm,top=35mm,bottom=30mm]{geometry}
  \setFancyHdr % To apply fancy header after geometry package is loaded
\fi

% Add spaces between paragraphs
%\setlength{\parskip}{0.5em}
% Ragged bottom avoids extra whitespaces between paragraphs
\raggedbottom
% To remove the excess top spacing for enumeration, list and description
%\usepackage{enumitem}
%\setlist[enumerate,itemize,description]{topsep=0em}

% *****************************************************************************
% ******************* Fonts (like different typewriter fonts etc.)*************

% Add `customfont' in the document class option to use this section

\ifsetCustomFont
  % Set your custom font here and use `customfont' in options. Leave empty to
  % load computer modern font (default LaTeX font).
  %\RequirePackage{helvet}

\usepackage{polyglossia}
\usepackage[autostyle=true]{csquotes}

\setmainlanguage[variant=mono]{greek}
\setotherlanguage{english}

\usepackage{amsmath}
\usepackage[]{unicode-math}
\setmainfont{XITS}
[    Extension = .otf,
   UprightFont = *-Regular,
      BoldFont = *-Bold,
    ItalicFont = *-Italic,
BoldItalicFont = *-BoldItalic,
]

\setmathfont{XITSMath-Regular}
[    Extension = .otf,
     BoldFont = XITSMath-Bold,
]

\newfontfamily\greekfont[Script=Greek]{XITS}
\newfontfamily\greekfontsf[Script=Greek]{XITS}
\newfontfamily\greekmathfont{XITSMath-Regular}


  % For use with XeLaTeX
  %  \setmainfont[
  %    Path              = ./libertine/opentype/,
  %    Extension         = .otf,
  %    UprightFont = LinLibertine_R,
  %    BoldFont = LinLibertine_RZ, % Linux Libertine O Regular Semibold
  %    ItalicFont = LinLibertine_RI,
  %    BoldItalicFont = LinLibertine_RZI, % Linux Libertine O Regular Semibold Italic
  %  ]
  %  {libertine}
  %  % load font from system font
  %  \newfontfamily\libertinesystemfont{Linux Libertine O}
\fi

% *****************************************************************************
% **************************** Custom Packages ********************************
% ************************* Algorithms and Pseudocode **************************

%\usepackage{algpseudocode}


% ********************Captions and Hyperreferencing / URL **********************

% Captions: This makes captions of figures use a boldfaced small font.
%\RequirePackage[small,bf]{caption}

\RequirePackage[labelsep=space, tableposition=top, textfont=it]{caption}
\renewcommand{\thefigure}{\thechapter.\arabic{figure}}
\renewcommand{\thetable}{\thechapter.\arabic{table}}
\renewcommand{\figurename}{Figure} %to support older versions of captions.sty

\usepackage{prettyref}
\newrefformat{ch}{Κεφάλαιο~\ref{#1}}
\newrefformat{app}{Παράρτημα~\ref{#1}}
\newrefformat{fig}{Σχήμα~\ref{#1}}
\newrefformat{tab}{Πίνακας~\ref{#1}}
\newrefformat{plt}{Γράφημα~\ref{#1}}
\newrefformat{eq}{σχέση~\ref{#1}}
\newrefformat{code}{Αλγόριθμο~\ref{#1}}

% *************************** Graphics and figures *****************************

%\usepackage{rotating}
%\usepackage{wrapfig}

% Uncomment the following two lines to force Latex to place the figure.
% Use [H] when including graphics. Note 'H' instead of 'h'0
%\usepackage{float}
%\restylefloat{figure}

% Subcaption package is also available in the sty folder you can use that by
% uncommenting the following line
% This is for people stuck with older versions of texlive
%\usepackage{sty/caption/subcaption}
\usepackage{subcaption}

% ********************************** Tables ************************************
\usepackage{booktabs} % For professional looking tables
\usepackage{multirow}
\usepackage{multicol}
\usepackage{longtable}
\usepackage{tabularx}

% ********************************** Plotting within LaTeX **********************
\usepackage{pgfplots}
\pgfplotsset{compat=1.8}
\usepackage{pgfplotstable}
\usepackage{filecontents}

\usepackage{tikz}
\usetikzlibrary{shapes, arrows, shapes.multipart}


% *********************************** SI Units *********************************
\usepackage[detect-all]{siunitx} % use this package module for SI units


% ******************************* Line Spacing *********************************
\usepackage{float}
% Choose linespacing as appropriate. Default is one-half line spacing as per the
% University guidelines

% \doublespacing
% \onehalfspacing
% \singlespacing


% ************************ Formatting / Footnote *******************************

% Don't break enumeration (etc.) across pages in an ugly manner (default 10000)
%\clubpenalty=500
%\widowpenalty=500

%\usepackage[perpage]{footmisc} %Range of footnote options


% *****************************************************************************
% *************************** Bibliography  and References ********************

%\usepackage{cleveref} %Referencing without need to explicitly state fig /table

% Add `custombib' in the document class option to use this section
\ifuseCustomBib
%   \RequirePackage[square, sort, numbers, authoryear]{natbib} % CustomBib

% If you would like to use biblatex for your reference management, as opposed to the default `natbibpackage` pass the option `custombib` in the document class. Comment out the previous line to make sure you don't load the natbib package. Uncomment the following lines and specify the location of references.bib file

\RequirePackage[backend=biber, style=numeric-comp,
				citestyle=numeric, sorting=nty,
				isbn=false, doi=true,
				url = false, natbib=false,
				defernumbers=true]{biblatex}
\addbibresource{References/references.bib} %Location of references.bib only for biblatex, Do not omit the .bib extension from the filename.

\fi

% changes the default name `Bibliography` -> `References'
%\renewcommand{\bibname}{References}


% ******************************************************************************
% ************************* User Defined Commands ******************************
% ******************************************************************************

% *********** To change the name of Table of Contents / LOF and LOT ************


%\renewcommand{\contentsname}{My List of Contents}
%\renewcommand{\listfigurename}{My List of Figures}
%\renewcommand{\listtablename}{Λίστα πινάκων}


% ********************** TOC depth and numbering depth *************************

\setcounter{secnumdepth}{2}
\setcounter{tocdepth}{2}


% ******************************* nomenclature *********************************

% Increase spacing between notation and text
\setlength{\nomlabelwidth}{1.5cm}

% To decrease the spacing between two consecutive notations
%\setlength{\nomitemsep}{-\parsep}

% To change the name of the Nomenclature section, uncomment the following line
\renewcommand{\nomname}{Αποδόσεις όρων}


% ********************************* Appendix ***********************************

% The default value of both \appendixtocname and \appendixpagename is `Appendices'. These names can all be changed via:

%\renewcommand{\appendixtocname}{List of appendices}
%\renewcommand{\appendixname}{Appndx}

% *********************** Configure Draft Mode **********************************

% Uncomment to disable figures in `draft'
%\setkeys{Gin}{draft=true}  % set draft to false to enable figures in `draft'

% These options are active only during the draft mode
% Default text is "Draft"
%\SetDraftText{DRAFT}

% Default Watermark location is top. Location (top/bottom)
%\SetDraftWMPosition{bottom}

% Draft Version - default is v1.0
%\SetDraftVersion{v1.1}

% Draft Text grayscale value (should be between 0-black and 1-white)
% Default value is 0.75
%\SetDraftGrayScale{0.8}


% ******************************** Todo Notes **********************************
%% Uncomment the following lines to have todonotes.

%\ifsetDraft
%	\usepackage[colorinlistoftodos]{todonotes}
%	\newcommand{\mynote}[1]{\todo[author=kks32,size=\small,inline,color=green!40]{#1}}
%\else
%	\newcommand{\mynote}[1]{}
%	\newcommand{\listoftodos}{}
%\fi

% Example todo: \mynote{Hey! I have a note}

% *****************************************************************************
% ******************* Better enumeration my MB*************
\usepackage{enumitem}


% ******************************** Typesetting Matlab Code **********************

\usepackage{bigfoot, textcomp, listings, xspace, fancyvrb}
\usepackage[numbered, framed, final]{matlab-prettifier}

\lstset{
  style              = Matlab-editor,
  basicstyle         = \mlttfamily,
  mlshowsectionrules = true,
  texcl=true,
  escapebegin=\begin{english},
  escapeend=\end{english},
  extendedchars =\true,
  aboveskip=20pt,
  belowskip=20pt
}


\usepackage[breakable]{tcolorbox}
\tcbuselibrary{listings}
\usepackage{algorithm2e, setspace, courier}
\tcbset{
  noparskip/.style={before={\par\smallskip\pagebreak[0]\parindent=0pt },
                    after={\par\smallskip}}
}

% redefine \VerbatimInput
\RecustomVerbatimCommand{\VerbatimInput}{VerbatimInput}%
{fontsize=\footnotesize,
 %
 frame=lines,  % top and bottom rule only
 framesep=2em, % separation between frame and text
 rulecolor=\color{Gray},
 %
 label=\fbox{\color{Black}results.txt},
 labelposition=topline
 %
% commandchars=\|\(\), % escape character and argument delimiters for
                      % commands within the verbatim
% commentchar=*        % comment character
}


\renewcommand{\lstlistingname}{Αλγόριθμος}
\renewcommand{\lstlistlistingname}{Λίστα αλγορίθμων}

\newcommand{\matlab}{\textsc{MATLAB}\textsuperscript{\textregistered}\xspace}
\newcommand{\inkscape}{\textit{Inkscape}\textsuperscript{\textcopyright}\xspace}

% ******************************** Macros used **********************************

\newcommand{\volume}{\ooalign{$V$\cr\raisebox{0.15em}{\kern0.08em\textendash}\cr}}

\makeatletter
\renewcommand{\@chapapp}{}% Not necessary...
\newenvironment{chapquote}[2][2em]
  {\setlength{\@tempdima}{#1}%
   \def\chapquote@author{#2}%
   \parshape 1 \@tempdima \dimexpr\textwidth-2\@tempdima\relax%
   \itshape}
  {\par\normalfont\hfill\textendash\ \chapquote@author\hspace*{\@tempdima}\par\bigskip}
\makeatother

\newcommand{\ITU}{İTÜ}
\newcommand{\TikZ}{Ti\textit{k}Z\xspace}

\newcommand{\ra}[1]{\renewcommand{\arraystretch}{#1}}

% ******************************** Miscellany ************************************
\definecolor{SchoolColor}{RGB}{11, 66, 104}
\usepackage{lettrine, adjustbox, threeparttable}
%\usepackage{pdflscape} %adds the attribute /Rotate in a landscape

%\addto\captionsenglish{\renewcommand{\chaptername}{Lecture}}
\makeatletter
\renewcommand{\@chapapp}{Κεφάλαιο}
\makeatother

\usepackage{cancel} % cancel terms out it math equations

\hyphenation{ρευ-στο-μη-χα-νι-κής}
\hyphenation{Ü-ni-ve-rsi-te-si}
\hyphenation{Μη-χα-νο-λό-γων}
\hyphenation{Πα-νε-πι-στή-μιο}
\hyphenation{Ε-κδό-σεις}